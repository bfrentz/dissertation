%
% Modified by Megan Patnott
% Last Change: Jan 18, 2013
%
%%%%%%%%%%%%%%%%%%%%%%%%%%%%%%%%%%%%%%%%%%%%%%%%%%%%%%%%%%%%%%%%%%%%%%%%
%
% Modified version of the sample_ndthesis.tex
% by Sameer Vijay
% Last Change: Wed Jul 27 2005 14:00 CEST
%
%%%%%%%%%%%%%%%%%%%%%%%%%%%%%%%%%%%%%%%%%%%%%%%%%%%%%%%%%%%%%%%%%%%%%%%%
%
% Sample Notre Dame Thesis/Dissertation
% Using Donald Peterson's ndthesis classfile
%
% Written by Jeff Squyres and Don Peterson
%
% Provided by the Information Technology Committee of
%   the Graduate Student Union%
%   http://www.gsu.nd.edu/
%
% Nothing in this document is serious except the format.  :-)
%
%%%%%%%%%%%%%%%%%%%%%%%%%%%%%%%%%%%%%%%%%%%%%%%%%%%%%%%%%%%%%%%%%%%%%%%%
% This is *not* a substitute for the documentation, which is included
% as a pdf file in the standard distribution, and can be obatined from
% the dtx file in the advanced distribution.
%%%%%%%%%%%%%%%%%%%%%%%%%%%%%%%%%%%%%%%%%%%%%%%%%%%%%%%%%%%%%%%%%%%%%%%%
%
% You should *also* have a ND formatting guide to ensure that you have
% all the relevant parts, put the captions in the right place, etc.
% Just because you have this wonderful style classfile doesn't mean
% that it removes *all* the formatting onus from you.  :-)
% Although be warned that the Graduate School has been known to let
% their official formatting guide get out of date. When in doubt,
% the Microsoft Word example seemed to be the only thing kept
% consistently up-to-date in 2013, and is probably the safest thing
% to consult.
%
% You should break all of this stuff up into separate files
% (at the very least, one chapter per file) and use the \include
% command, as has been done here for chapters 1 and 2 and the appendix.
% There is also an \input command, but \include is more commonly used to
% import chapters in books and dissertations. For the differences between these
% two commands, see, e.g., 
% http://web.science.mq.edu.au/~rdale/resources/writingnotes/latexstruct.html
% or http://tex.stackexchange.com/questions/246/when-should-i-use-input-vs-include.
%
% If you compile from the command line, note that you should also have 
% a good Makefile; one that invokes LaTeX as many times as necessary 
% (up to 4) and bibtex if necessary.
%
% If you use an editor that allows you to compile from within the
% program, note that you will need to compile up to four times. Also,
% we recommend that you use pdflatex (sometimes displayed as
% LaTeX => PDF) to compile directly to pdf.
%
% If you have any suggestions, comments, questions, please send e-mail
% to: dteditor@nd.edu
%
%%%%%%%%%%%%%%%%%%%%%%%%%%%%%%%%%%%%%%%%%%%%%%%%%%%%%%%%%%%%%%%%%%%%%%%%

\documentclass[numrefs,sort&compress,review]{nddiss2e}
% One of the options draft, review, final must be chosen.
% One of the options textrefs or numrefs should be chosen
% to specify if you want numerical or ``author-date''
% style citations.
% Other available options are:
% 10pt/11pt/12pt (available with draft only)
% twoadvisors
% noinfo (should be used when you compile the final time
%         for formal submission)
% sort (sorts multiple citations in the order that they're
%       listed in the bibliography)
% compress (compresses numerical citations, e.g. [1,2,3]
%           becomes [1-3]; has no effect when used with
%           the textrefs option)
% sort&compress (sorts and compresses numerical citations;
%           is identical to sort when used with textrefs)

\usepackage{graphicx}% Include figure files
\usepackage{amssymb}
\usepackage{amsmath}
\usepackage{mhchem}
\usepackage{revsymb}
\usepackage{bm}
\usepackage{realboxes}
\usepackage{morefloats}
\usepackage[flushleft]{threeparttable} 
\usepackage{rotating}
\usepackage{floatpag}
%\usepackage{float}
%\pagestyle{plain}
%\usepackage{subfig}
\usepackage{subcaption}
\usepackage{mwe}
\usepackage{enumitem}
\usepackage{gensymb}
\usepackage{graphicx}
\usepackage{topcapt}
\usepackage{booktabs}
\usepackage{commath}
\usepackage{units}
\usepackage{makecell}
\usepackage{adjustbox}
\usepackage{listings}
\lstset{language=C}
\lstset{breaklines}
\usepackage{xcolor}
 
\definecolor{codegreen}{rgb}{0,0.6,0}
\definecolor{codegray}{rgb}{0.5,0.5,0.5}
\definecolor{codepurple}{rgb}{0.58,0,0.82}
\definecolor{backcolour}{rgb}{0.95,0.95,0.92}

\lstdefinestyle{mystyle}{
    backgroundcolor=\color{backcolour},   
    commentstyle=\color{codegreen},
    keywordstyle=\color{magenta},
    numberstyle=\tiny\color{codegray},
    stringstyle=\color{codepurple},
    basicstyle=\ttfamily\footnotesize,
    breakatwhitespace=false,         
    breaklines=true,                 
    captionpos=b,                    
    keepspaces=true,                 
    numbers=left,                    
    numbersep=5pt,                  
    showspaces=false,                
    showstringspaces=false,
    showtabs=false,                  
    tabsize=2
}
 
\lstset{style=mystyle}
\usepackage{pdfpages}
\usepackage{bibentry}


\begin{document}

\frontmatter % All the items before the first chapter go in ``frontmatter''

% Titles may be 1-4 lines long. If your title is longer than 4 lines,
% the class file may have difficulty formatting the title page.
% Line-breaks in the title have to be protected with `\protect`.
\title{An investigation of the astrophysically important $^{14}$N$\left( p,\gamma \right) ^{15}$O reaction}
% TITLE OF WORK. It must be in all caps, and ensuring this is your
 % responsiblity.
\author{Bryce Alan Frentz}
\work{Dissertation} % or \work{Thesis}
\degaward{Doctor of Philosophy} % or 
%\degaward{Master of Science \\ in \\ Subject}
\advisor{Ani Aprahamian}
\department{Physics}

\maketitle
%%%%%%%%%%%%%%%%%%%%%%%%%%%%%%%%%%%%%%%%%%%%%%%%%%%%%%%%%%%%%%%%%%%%%%%%
%
% Front stuff
%
%%%%%%%%%%%%%%%%%%%%%%%%%%%%%%%%%%%%%%%%%%%%%%%%%%%%%%%%%%%%%%%%%%%%%%%%

% You must either set the copyright information or put your work in the public domain.
\copyrightholder{Bryce Frentz} % See template or documentation for
\copyrightyear{2021}           % other copyright options.
\copyrightlicense{CC-BY-4.0}
\makecopyright

% An abstract is optional for a mster's thesis, and required for a doctoral dissertation.
\begin{abstract}
  
The CNO cycle is the main energy source in massive stars during their hydrogen burning phase, but it also contributes to the energy production in our sun at the $\sim$ 1$\%$ level. The $^{14}$N$(p,\gamma)^{15}$O reaction is the slowest reaction in the cycle and as such determines not only the energy production rate but also the rate by which CNO neutrinos contribute to the solar neutrino flux. The CNO neutrinos come primarily from the $\beta$ decay of $^{15}$O and to a lesser extent from the decay of $^{13}$N. $^{15}$O nuclei are produced via the $^{14}$N$(p,\gamma)^{15}$O reaction in the CNO cycle in stellar cores. Recent measurements of CNO neutrinos from the decay of $^{15}$O at the Borexino detector provided the first direct measurement of CNO neutrinos, indicating a higher rate than previously expected. These findings provided new, independent information about the metalicity of the solar core.

There are still considerable uncertainties in the rate for the $^{14}$N$(p,\gamma)^{15}$O reaction at solar temperature conditions coming from nuclear sources, namely the low-energy cross section data for the transitions to the ground state and 6.79 MeV excited state and the lifetime of the subthreshold excited state in $^{15}$O at 6.79 MeV. 

In this work, the result of a low energy $^{14}$N$(p,\gamma)^{15}$O reaction measurement taken at the CASPAR underground accelerator facility, with a JN Van de Graaff accelerator, is presented. The measurement detailed here aims to find a resolution to existing discrepancies in the data sets and moving towards a better understanding of the low-energy behavior of the $^{14}$N$(p,\gamma)^{15}$O cross-section. The data span proton energies between $E_{p}$ = 0.27 and 1.07 MeV. The resulting cross sections close a gap in the overall data that existed between other low-energy measurements, like from the LUNA facility, and high-energy measurements. Sputtered ZrN targets were utilized and the de-excitation $\gamma$ rays were detected at 55$\degree$ with a 130$\%$ efficiency high-purity germanium detector. 

This work also addresses the predominant uncertainty coming from the lifetime of the excited state at $E_{x}$ = 6.79 Mev. Previous measurements of this state's lifetime are significantly discrepant. The second half of this work was a measurement of this state, as well as the excited states in $^{15}$O at $E_{x}$ = 5.18 MeV and $E_{x}$ = 6.17 MeV. The excited $^{15}$O nuclei were populated via the $^{14}$N$(p,\gamma)^{15}$O reaction at proton energies of $E_{p} = 1.02$ MeV and $E_{p} = 1.57$ MeV. The lifetimes have been determined by the Doppler-Shift Attenuation Method (DSAM) with three separate, nitrogen-implanted targets of Mo, Ta, and W backing. There lifetimes were obtained from the weighted average of the three individual measurements, allowing us to account for systematic differences between the backing materials. For the 6.79 MeV state, we obtained a $\tau = 0.6 \pm 0.4$ fs. To provide cross-validation of our method, the lifetimes of the states at 5.18 MeV and 6.17 MeV to be $\tau = 7.5 \pm 3.0$ and $\tau = 0.7 \pm 0.5$ fs, respectively, in good agreement with previous measurements. 

Ultimately, a multichannel $R$-matrix analysis was performed on the all of the present data and was used to extrapolate the astrophysical $S$ factors of the ground state and the 6.79 MeV transitions to low energies. The present extrapolated zero-energy $S$ factors are $S_{6.79} (0) = 1.24 \pm 0.09$ keV b, $S_{6.17} (0) = 0.12 \pm 0.05$ keV b, and $S_{g.s.} (0) = 0.33_{-0.08}^{+0.16}$ keV b.

The value for the $S_{6.79}(0)$ overlaps well with recently reported measurements. Overall, this indicates that this transition's extrapolation is quite robust to the addition of new data. For the ground state zero-energy extrapolation, $S_{g.s.}(0)$, is higher than most found in literature, but still lower than the absolute highest reported values.

The approach taken here for the study of the $^{14}$N$(p,\gamma)^{15}$O reaction has succeeded in combining both the efforts of an improved lifetime measurement and new cross section data in underground accelerator experiments. The combination of these two complementary measurements allows us to suggest an enhancement to the zero-energy $S$-factor of the $^{14}$N$(p,\gamma)^{15}$O reaction, ultimately supporting the higher rate observed in the recent Borexino neutrino measurements. However, following the $R$-matrix analysis, consistent discrepancies between measured data and fits, both past and present, are identified and recommendations for future study are made.
  
\end{abstract}

% A dedication is optional.
\renewcommand{\dedicationname}{DEDICATION}

\begin{dedication}
 This one's for me.
\end{dedication}

% These are required, and must be in this order.
\tableofcontents
\listoffigures
\listoftables

% A preface is optional.
%\begin{preface}
  
%When nothing is done, nothing is left undone. - Lao Tzu
  
% The purpose of a storyteller is not to tell someone how to think, but to give someone questions to think upon. - Brandon Sanderson
  
%  When numbers acquire the significance of language they acquire the power to do all of the things that language can do. Describe power history success failure victory defeat character grace to become fiction, drama, and poetry. 
  
%\end{preface}

% It's hard to tell from the information available from the Graduate
% School in Spring 2013 whether or not an acknowledgements section is optional.
\begin{acknowledge}
Though my name is the one on the cover, there were tons of people involved in my journey and they deserve an inordinate amount of thanks.

% Thank you to Ani Aprahamian and Michael Weischer for the indelible mark you've left on my life and work. Particularly, also, thank you for teaching me about effective leadership and management. Joachim G{\"o}rres and his wisdom were integral in completing this work. 

% To Khachatur Manukyan, Wanpeng Tan, James DeBoer, Patrick O'Malley, Dan Robertson, Ed Stech, and Janet Weikel: thank you all for your constant friendship and for the myriad ways in which you've all helped me over the years. 

% To my long-standing office mates, Qian Liu and Matt Hall, thank you so much for the friendship and fun we've had together, from teaching each other our languages to being running partners and everything inbetween, including sharing burritos. I couldn't have made it through without you.

% To my friends at Notre Dame, thank you for all of the wonderful times going to dinner and movies, talking shop and being down for whatever.

% Kevin and Christina

% To my friends in Louisville, thank you all for the constant 

% Inspiration? Figures from history and musicians?

% To my family

% To my dog, Rudy,

% And finally, I need to acknowledge my substantial debt once more to the amazing Laura, for being so many things over the years - girlfriend, best friend, walking partner, constructive critic, and, at long last, wife. Throughout my endeavors, I've asked for your fortidude and forbearance, and you have always given it -- sometimes reluctantly, but always with love. Thank you, Laura, for being my Cecilia and my Hobbes. Thank you for being my wife, partner, and friend.

\end{acknowledge}

% A symbols section is optional.
%\begin{symbols}
%  \sym{\mathcal{F}}{sighting frequency of Gnus about campus}
%  \sym{p}{student population}
%  \sym{f}{type of food available}
%  \sym{d}{day of week}
%  \sym{c}{speed of light}
%  \sym{m}{mass}
%  \sym{e}{elementary charge}
%  \sym{a,b}{miscellaneous constants}  
%  \sym{E}{energy}  
%\end{symbols}

\mainmatter
% Place the text body here.
%\include{chapter-one}
%Begin each chapter with \chapter{Title}. Both the thesis title and
%chapter titles should match in style.

%
% An unnumbered chapter (features)
%
%\unnumchapter{Features of Formatting in This Example File}
%% The \unnumchapter command allows you to include an unnumbered chapter as part of
%% the main text before Chapter 1. It will appear in your table of contents, and you
%% should have at most one such chapter (although nothing in the class file will
%% prevent you from creating more).
%
%% The usual \cite{} command is also available, and should work as expected.
%This \verb+chapter+ has been added to the original sample file to highlight the
%various features with the formatting that conforms to the Graduate school
%guidelines --- whether obtained due to the use of \nddiss\/ class file or just
%plain good practice.
%\begin{itemize}
%\item An important note on line-breaks via \verb+\\+ in titles: the
%  titles of the thesis as well as chapters and table captions use
%  \verb+\MakeTextUppercase{}+ from the \verb+textcase+ package.  Due
%  to the nature of the \verb+center+ environment, any line-breaks
%  introduced in titles and captions should be protected, as in
%  \verb+\protect\\+.
%  To preserve the case in titles and captions, use, e.g.,
%  \verb+\NoCaseChange{Gnus}+.
%\item In the \emph{dedication}, the title name has been modified. So, you know
%how to and that it can be done.
%\item The entries in the \emph{List of figures} and \emph{List of Tables} are
%single-spaced themselves but are double-spaced from the other.
%\item The table captions are not in all CAPS as well for the reason mentioned
%above.
%\item Appropriate space is left between the \verb+Table xx+ and its
%corresponding caption (which is double-spaced itself) as in table \ref{tbl:bogus1}.
%\item Tables look much better without the vertical lines (good practice).
%\item There is double-spacing between the table entries but single-spacing
%within the entry.
%\item The chapter (see Chapter \ref{chap:golfing}) or section titles are
%double-spaced as mentioned in the guidelines.
%\item There is a \verb+subsubsection+ present (eg. section \ref{sec:data}) and
%is properly formatted in the TOC.
%\item Sections deeper than \verb+subsubsection+ should not appear in the TOC.
%\item Table \ref{tbl:defs} is an example of the use of \textsf{landscape}
%environment in which a normal table is formatted in a \emph{landscape} mode.
%\item The \textsf{longtable} environment is used in Tables \ref{tbl:votes} and
%\ref{tbl:rotated-rankings}, in normal and \verb+landscape+ mode, respectively. The
%table captions are formatted properly in both cases.
%\item In the table \ref{tbl:votes}, the \verb+footnote+ in the table header 
%does not appear at all. This is not an error of the \nddiss\/ class but of the
%\textsf{longtable} package.
%\item An example of citing a website is shown in the bibliography (see
%\citep{gairley2000}) which is formatted using the \verb+nddiss2e.bst+
%citation style file.
%\item A bit of information on the \nddiss\/ class file and the typesetting program
%used is included in a box on the last page of the thesis.
%\item Footnotes should space properly.
%\item Items in \verb+itemize+, \verb+enumerate+, and \verb+description+ environment
%should automatically single-space within an item, but double space between items.
%\end{itemize}

%
% Chapter 1
%

\include{chapter1}


%
% Chapter 2
%

%
% Modified by Megan Patnott
% Last Change: Jan 18, 2013
%
%%%%%%%%%%%%%%%%%%%%%%%%%%%%%%%%%%%%%%%%%%%%%%%%%%%%%%%%%%%%%%%%%%%%%%%%
%
% Modified by Bryce Frentz
% Last Change: 2018
%
%%%%%%%%%%%%%%%%%%%%%%%%%%%%%%%%%%%%%%%%%%%%%%%%%%%%%%%%%%%%%%%%%%%%%%%%
%
% Sample Notre Dame Thesis/Dissertation
% Using Donald Peterson's ndthesis classfile
%
% Written by Jeff Squyres and Don Peterson
%
% Provided by the Information Technology Committee of
%   the Graduate Student Union
%   http://www.gsu.nd.edu/
%
% Nothing in this document is serious except the format.  :-)
%
% If you have any suggestions, comments, questions, please send e-mail
% to: ndthesis@gsu.nd.edu
%
%%%%%%%%%%%%%%%%%%%%%%%%%%%%%%%%%%%%%%%%%%%%%%%%%%%%%%%%%%%%%%%%%%%%%%%%

%
% Chapter 2
%

\chapter{Experimental setup and procedures}
\label{chap: experiment}

\section{Introduction}

This chapter is included to highlight some techniques and equipment commonly used in low-energy nuclear physics, with an emphasis on those used in this work. The discussion will be presented from the lens of the general to the specific, introducing general concepts for beam production and transport, acceleration systems, and radiation detection. Following the discussion of the scientific principles, I will detail the specific and the specifics of the equipment employed in this collective work. 


\section{Accelerators, beam production, and radiation detection}
\label{sec: accelerators, beamline, and detectors}



\subsection{Van de Graaff accelerators}
\label{sec: vdg accelerators}

First and foremost, low-energy nuclear physics (generally understood to be the regime in which incident kinetic energies for reactions are below 1 GeV, often even as low as sub-MeV) requires particle acceleration systems in order to bring beams and targets together. The most common techniques for accelerating ions are 1) to use strong electrostatic fields for a straight-line acceleration, such as in the Van de Graaff accelerator, or 2)  by using a combination of electric and magnetic fields to accelerate the particles in cyclotron motion, aptly named a cyclotron. Van de Graaff accelerators are the most prominent of all accelerators, though, and are the type used in this work. 

The basic idea of a Van de Graaff accelerator is that a beam of ions produced in an ion source (Section \ref{sec: ion sources}) are manipulated by additional electromagnetic elements (Section \ref{sec: beamline}) into an area of high electric potential. This high voltage is produced and maintained by continuously transporting positive charge from ground to a Faraday cage with a field free interior (referred to as the terminal, with voltage $V_{T}$). Upon exiting the accelerator, the ions have energy

\begin{equation}
E_{ion} = qV_{T} = \dfrac{1}{2} m_{ion} v_{ion}^{2}
\label{eqn: accelerator}
\end{equation}

\noindent where $V_{T}$ again is the terminal voltage and $q$ is the charge state of the ion going through the accelerator, $m_{ion}$ is the mass of the ion, and $v_{ion}$ is the ion's velocity. When the terminal voltage is in MV, the beam energy is then provided in MeV. 

The charge is delivered to the terminal via a mechanical delivery system. The 5U Sta. Ana accelerator at the NSL uses four Pelletron chains cite{Herb1974}, each with its own power supply 


\subsection{Ion sources}
\label{sec: ion sources}



\subsection{Beam transport}
\label{sec: beamline}



\subsection{High-purity Germanium detectors}
\label{sec: HPGe detectors}


\section{Cross-section measurements}
\label{sec: cs experiment}



This set of data was taken over the course of five* separate experiments. The first occurred at the University of Notre Dame's Nuclear Science Laboratory (NSL) in January of 2018 and covered the proton energy range of E$_{p}$ = 800 - 1200 keV. The experiment was then continued at the Compact Accelerator System for Performing Nuclear Astrophysics (CASPAR) facility at the Sanford Underground Research Facility located in Lead, South Dakota in three increments: February 2018, May 2018, and August / September 2018. These measurements covered the energy range from E$_{p}$ = 270 - 1200 keV, in order to measure the $^{14}$N$\left( p,\gamma \right) ^{15}$O reaction cross-section to compare the performance of the CASPAR facility to an above-ground laboratory. Finally, in MONTH TIME DATE THING, the final experiment was completed at the NSL, focusing on obtaining the lifetime of the 6793 keV state in $^{15}$O.

\subsection{Measurement at Notre Dame}

\subsection{Measurment at CASPAR}


\section{Lifetime measurement}
\label{sec: lifetime experiment}

\subsection{Target production}

\subsection{Measurement at Notre Dame}




% % uncomment the following lines,
% if using chapter-wise bibliography
%
% \bibliographystyle{ndnatbib}
% \bibliography{example}


%
% Chapter 3
%

%
% Modified by Megan Patnott
% Last Change: Jan 18, 2013
%
%%%%%%%%%%%%%%%%%%%%%%%%%%%%%%%%%%%%%%%%%%%%%%%%%%%%%%%%%%%%%%%%%%%%%%%%
%
% Modified by Bryce Frentz
% Last Change: 2018
%
%%%%%%%%%%%%%%%%%%%%%%%%%%%%%%%%%%%%%%%%%%%%%%%%%%%%%%%%%%%%%%%%%%%%%%%%
%
% Sample Notre Dame Thesis/Dissertation
% Using Donald Peterson's ndthesis classfile
%
% Written by Jeff Squyres and Don Peterson
%
% Provided by the Information Technology Committee of
%   the Graduate Student Union
%   http://www.gsu.nd.edu/
%
% Nothing in this document is serious except the format.  :-)
%
% If you have any suggestions, comments, questions, please send e-mail
% to: ndthesis@gsu.nd.edu
%
%%%%%%%%%%%%%%%%%%%%%%%%%%%%%%%%%%%%%%%%%%%%%%%%%%%%%%%%%%%%%%%%%%%%%%%%

%
% Chapter 3
%

\chapter{Cross section data Reduction and analysis}
\label{chap: data}

\section{Introduction}

In aggregate, the data taken at both the CASPAR and NSL experiments consists of $\gamma$-ray energy data from the $^{14}$N$\left( p,\gamma \right) ^{15}$O reaction, observed with a single, 130\% HPGe detector placed at $55^{\degree}$ relative to the beam direction. These data were collected for reactions over the combined proton energy range of 270 - 1200 keV. The primary interest of these experiments was monitoring the R/DC$\rightarrow$GS transition and the R/DC$\rightarrow$6.79 MeV + 6.79 MeV$\rightarrow$GS transition sequence. As such, the energies of the concerned photons ranged in energy from $\sim$600 keV up to $\sim$8500 keV. This chapter details the processes by which this data is gathered and turned into an experimental cross section.

\section{Angular corrections}
\label{sec: angularCorrections}

The angular distribution of a cross section, $W$, can be described by

\begin{equation}
W_{\text{exp}} = a_{0} \left(1 + \sum_{i = 1}^{n} a_{i} Q_{i} P_{i} ( \cos (\theta) )    \right)
\end{equation}

\noindent where $a_{i}$ are the angular distribution coefficients, $Q_{i}$ are correction factors due to the finite size of a given detector, and the $P_{i} ( \cos (\theta) )$ are the Legendre polynomials of order $i$. For the conditions of this work, odd numbered terms as well as those of order higher than 2 give negligible contribution to the angular distribution. Therefore, the resulting angular distribution of this reaction is of the form

\begin{equation}
W_{\text{exp}} = a_{0} + a_{2} Q_{2} P_{2} ( \cos (\theta) ).
\end{equation}

Experimentally, to address any effects arising from an angular distribution of this form, the detector was placed at $55^{\degree}$. This is the zero of the 2nd order Legendre polynomial, thereby minimizing any effects on the cross section arising from the detector's angle. Simultaneously, this means that no correction of the data is necessary.


\section{Energy calibration}
\label{sec: energy calibration}

It is important to establish the true relationship between $\gamma$-rays of different energies incident in a detector and the signal they produce during data acquisition. This connection is determined by calibrating the system with $\gamma$-rays of well-known energy from room background, given radioactive sources, like $^{137}$Cs or $^{60}$Co, and well-studied nuclear reactions, like $^{27}$Al($p, \gamma$)$^{28}$Si. The reactions are also used in calibration because no natural sources of radioactivity provide $\gamma$'s with energy higher than 3.6 MeV, whereas the $^{27}$Al($p, \gamma$)$^{28}$Si reaction provides $\gamma$ ray energies up to 10.7 MeV, ensuring that the detector is well calibrated over the entire energy range for $\gamma$'s that will be seen in the $^{14}$N$\left( p,\gamma \right) ^{15}$O reaction. The exact relationship between the the channel number in the acquisition system analog-to-digital converter (ADC) and the incident photon energy, $E_{\gamma}$, is characterized by the standard linear relationship

\begin{equation}
E_{\gamma} = m \times \text{Channel} + b
\end{equation}

\noindent where $m$ is simply the slope and $b$ the offset of the fit. For a HPGe detector, a linear relationship is sufficient and appropriate to describe the ADC response. However, this process was redone for every phase of the experiment because slightly different gains applied to the ADC's and signal amplifiers provide a different relationship in the electronics. Therefore, despite using the same detector, each phase of the experiment required its own energy calibration, an example of which is shown in Fig.\ \


\begin{figure}
\centering
\includegraphics[width=0.8\linewidth]{figures/expSetup.png}
\label{fig: energyCalibration}
\caption{Energy calibration curve for the HPGe detector taken at CASPAR. The calibration incorporated natural background, radioactive sources, and the products from the $^{27}$Al($p, \gamma$)$^{28}$Si reaction. }
\end{figure}




\section{Efficiency}
\label{sec: efficiency}

For this experiment, both the total efficiency, $\eta^{tot}$, and the full-energy peak (FEP) efficiency, $\eta^{fep}$, were required. The total efficiency is the probability that the $\gamma$ ray enters and deposits any amount of energy within the detector, while, on the other hand, the FEP efficiency is the probability that the full energy of an emitted $\gamma$ ray will be deposited within the detector. As both efficiency types are dependent on the physical geometry of the system, they are determined for each experimental setup in turn. 

\subsection{Total efficiency}

The total efficiency of a detector / source geometry is the probability that a given photon from the source will enter and deposit any amount of energy within the detector. For extremely simple geometries, the total efficiency can be calculated following the approach laid out in Debertin and Helmer cite{DebertinHelmerBook},

\begin{equation}
\eta^{tot} = \dfrac{1}{4\pi} \int \left(1 - e^{-\mu x} \right) d\Omega
\end{equation}

\noindent where $\mu$ is the energy and absorber dependent attenuation coefficient, x is the position from the detector face, and the integration takes place over the solid angle of the detector from the front to the back. This formulation makes it abundantly clear that $\eta^{tot}$ is so highly dependent on the geometry of the setup. 

However, in practice, there is no analytic formulation of the total efficiency because one needs to account for any potential scattering of $\gamma$-rays off of any surrounding material, like the shielding or target chamber, for example. Additionally, when measuring the total efficiency for a given setup, the presence of multiple decays from physical sources provides additional complications. Therefore, most commonly the total efficiency is determined using single-line radioactive sources, like $^{137}$Cs. In this case, when accounting for the ever-present background, the total efficiency is the total number of counts in a spectrum divided by the total number of decays occurring during the measurement time, which can be easily calculated with the radioactive decay law

\begin{equation}
N(t) = N_{0} e^{- \lambda t} \hfill A(t) = A_{0} e^{- \lambda t},
\end{equation}

\noindent where $N(t)$ is the number of nuclei remaining after time $t$, $N_{0}$ is the initial number of radioactive nuclei, $\lambda$ is the radioactive decay constant for a given nucleus, $A(t)$ is the activity of the source at time $t$ compared to the activity at time $t=0$ of $A_{0}$. The decay constant, $\lambda$, can be calculated from either the nuclear lifetime, $\tau$, or the decay half-life, $t_{1/2}$, as

\begin{equation}
\lambda = \dfrac{1}{\tau} \hfill \lambda = \dfrac{\text{ln}(2)}{t_{1/2}}.
\end{equation}

\noindent This is how the total efficiency for the detector systems was determined at 662 keV, the $\gamma$ energy from the decay of $^{137}$Cs.



\begin{figure}
\centering
\includegraphics[width=\linewidth]{figures/shieldingSimulation.png}
\label{fig: simulatedSetup}
\caption{Simulation of the experimental setup at CASPAR, including the target chamber (in gold), water cooling (in light blue), HPGe detector (in dark blue), and lead bricks for shielding (in gray). Both pictures are of the same experimental setup from different angles to convey the layout. Simulating this setup is necessary to determine the total efficiency of the experimental setup. In this determination, a single-line $\gamma$ source is placed at the target location and its decays simulated for a variety of energies, allowing a determination of the total efficiency at each.}
\end{figure}


\begin{figure}
\centering
\includegraphics[width=0.8\linewidth]{figures/shieldingPicture.jpg}
\label{fig: actualSetup}
\caption{A picture of the experimental setup at CASPAR, showing the lead shielding around the detector and target chamber. This figure is provided for comparison to the simulated setup to prove its accuracy. }
\end{figure} 

To extend $\eta^{tot}$ to the whole range of energies relevant to the experiments, the experimental setup (including the target chamber, water cooling, and shielding) was built and simulated in Geant4 \cite{Agostinelli2003}, an example of the simulation for the CASPAR setup is shown in Fig.\ \ref{fig: simulatedSetup} where the actual setup is given in Fig.\ \ref{fig: actualSetup} for comparison. Then, a single-emission source was placed at the target location and simulated 10,000,000 decays, recording the deposited energy inside the detector. The energy of the source was changed and simulated at energies of 662, 1300, 2000, 3000, 4000, 6000, 7000, 8000, and 10000 keV to cover the entire range of $\gamma$ ray energies present in the $^{14}$N$\left( p,\gamma \right) ^{15}$O reaction. For each of the different decay energies, the spectrum of recorded events in the detector was analyzed in the exact same way as the actual $^{137}$Cs data to produce the total efficiency at each energy. This simulated data was used in conjunction with the measured $^{137}$Cs total efficiency data point to determine an accurate curve of the total efficiency across all experimentally relevant energies, shown in Fig.\ \ref{fig: totalEfficiency}.

\begin{figure}
\centering
\includegraphics[width=\linewidth]{figures/totalEfficiency.png}
\label{fig: totalEfficiency}
\caption{The total efficiency of the HPGe in the CASPAR setup across all relevant energies. The points from the Geant4 simulation are taken raw without scaling, showing the efficacy of this technique and accuracy of the simulation.  }
\end{figure}



\subsection{Full energy peak efficiency}


\section{Summing corrections}
\label{sec: summing}

\section{Target characterization}
\label{sec: target}

\section{Cross-section determination}
\label{sec: cross-section}





% % uncomment the following lines,
% if using chapter-wise bibliography
%
% \bibliographystyle{ndnatbib}
% \bibliography{example}


%
% Chapter 4
%

%
% Modified by Megan Patnott
% Last Change: Jan 18, 2013
%
%%%%%%%%%%%%%%%%%%%%%%%%%%%%%%%%%%%%%%%%%%%%%%%%%%%%%%%%%%%%%%%%%%%%%%%%
%
% Modified by Sameer Vijay
% Last Change: Tue Jul 26 2005 13:00 CEST
%
%%%%%%%%%%%%%%%%%%%%%%%%%%%%%%%%%%%%%%%%%%%%%%%%%%%%%%%%%%%%%%%%%%%%%%%%
%
% Sample Notre Dame Thesis/Dissertation
% Using Donald Peterson's ndthesis classfile
%
% Written by Jeff Squyres and Don Peterson
%
% Provided by the Information Technology Committee of
%   the Graduate Student Union
%   http://www.gsu.nd.edu/
%
% Nothing in this document is serious except the format.  :-)
%
% If you have any suggestions, comments, questions, please send e-mail
% to: ndthesis@gsu.nd.edu
%
%%%%%%%%%%%%%%%%%%%%%%%%%%%%%%%%%%%%%%%%%%%%%%%%%%%%%%%%%%%%%%%%%%%%%%%%


%
% Chapter 4
%


\chapter{Measured Lifetimes}
\label{chap: lifetime}

\section{Determining a nuclear lifetime}
\label{sec: lifetime simulation}


\section{Lifetime of the 5.18 MeV state in $^{15}$O}
\label{sec: lifetime518}


\section{Lifetime of the 6.17 MeV state in $^{15}$O}
\label{sec: lifetime617}


\section{Lifetime of the 6.79 MeV state in $^{15}$O}
\label{sec: lifetime679}

% % uncomment the following lines,
% if using chapter-wise bibliography
%
% \bibliographystyle{ndnatbib}
% \bibliography{example}


%
% Chapter 5
%

%
% Modified by Megan Patnott
% Last Change: Jan 18, 2013
%
%%%%%%%%%%%%%%%%%%%%%%%%%%%%%%%%%%%%%%%%%%%%%%%%%%%%%%%%%%%%%%%%%%%%%%%%
%
% Modified by Sameer Vijay
% Last Change: Tue Jul 26 2005 13:00 CEST
%
%%%%%%%%%%%%%%%%%%%%%%%%%%%%%%%%%%%%%%%%%%%%%%%%%%%%%%%%%%%%%%%%%%%%%%%%
%
% Sample Notre Dame Thesis/Dissertation
% Using Donald Peterson's ndthesis classfile
%
% Written by Jeff Squyres and Don Peterson
%
% Provided by the Information Technology Committee of
%   the Graduate Student Union
%   http://www.gsu.nd.edu/
%
% Nothing in this document is serious except the format.  :-)
%
% If you have any suggestions, comments, questions, please send e-mail
% to: ndthesis@gsu.nd.edu
%
%%%%%%%%%%%%%%%%%%%%%%%%%%%%%%%%%%%%%%%%%%%%%%%%%%%%%%%%%%%%%%%%%%%%%%%%


%
% Chapter 5
%


\chapter{R-matrix analysis}
\label{chap: r-matrix}

An $R$-matrix analysis was performed to analyze the effects of the new lifetime and cross sections measurements described earlier. Such an analysis can be used to transform the cross sections into $S$-factors, which removes the Coulomb repulsion component of the cross-section and provides higher fidelity extrapolations to astrophysical energies. The $^{14}$N$(p,\gamma)^{15}$O system is ideal for $R$-matrix analysis because it is comprised of light nuclei with a discrete level structure at relatively low energies. While there are many programs available for performing $R$-matrix fits, we use the \texttt{AZURE2} code \cite{Azuma2010}, which was also used for similar, previous measurements \cite{Li2016}. 

\section{Impact of new lifetimes}
\label{sec: lifetime fit}

Following the analysis of \citet{Li2016}, the width of the 6.79 MeV state in $^{15}$O  was previously the largest source of uncertainty in the low-energy extrapolations of the cross section. A set of $R$-matrix fits was employed to explore the impact of our newly measured lifetimes. To isolate the effects of our measurement, we selected discrete lifetimes within our uncertainty range for the lifetime of the 6.79 MeV state in $^{15}$O, converted them to their equivalent radiative width, and used those level width values throughout the fit and extrapolation. This collection of fits, therefore, serves primarily as an illustration of the ways in which these new lifetime measurements impact the low energy extrapolations of the cross section.

In the fits presented here, the information about the levels was taken from  \citet{Ajzenberg-Selove1991} or \citet{Daigle2016} where updated. A channel radius of 5.5 fm was adopted for this work, which matches the analyses done by Refs.~\cite{Adelberger2011, Li2016, Wagner2018}. Information about the levels and their parameters as used in \texttt{AZURE2} are contained in Table~\ref{table: fitParams}. 


\begin{table*}[]
\thisfloatpagestyle{plain}
\caption{PARAMETERS USED IN THE $R$-MATRIX FITS FOR EXPLORING THE IMPACT OF NEWLY MEASURED LIFETIMES}
\begin{center}
\begin{threeparttable}
\begin{tabular}{c  c  c  c  c  c  c}
\toprule
$E_x$ (Ref.~\cite{Ajzenberg-Selove1991}) &   $E_x$ (fit) & $J^\pi$ & Channel & l & s & ANC (fm$^{-1/2}$) / $\Gamma$ (eV)\\ 
\midrule
0.0 & 0.0	& 1/2$^-$ &	$^{14}$N+p &	1&	1/2&	{0.23}\\
	&	&	    &    $^{14}$N+p &	1&	3/2&	{7.4} \\
%5.183(1) & \textbf{5.183}&	1/2$^+$&	$^{14}$N+p&	0&	1/2&	\textbf{0.33}\\
%	&	&		    &$^{15}$O+$\gamma_{0.00}$  &	E1&	1/2&	\textbf{0.0784}\\
%5.2409(3) & \textbf{5.2409}&	5/2$^+$&	$^{14}$N+p &	2&	1/2&	\textbf{0.23}\\
%                &	&		  & $^{14}$N+p &	2&	3/2&	\textbf{0.24}\\
%	 			&	&	 &$^{15}$O+$\gamma_{0.00}$	&M2&	1/2&	\textbf{0.0002}\\
%6.1763(17) & \textbf{6.1763}&	3/2$^-$& $^{14}$N+p &	1&	1/2&	\textbf{0.47}\\
%				&	&	& $^{14}$N+p	&1	&3/2	&\textbf{0.53}\\
%				&	&	&$^{15}$O+$\gamma_{0.00}$	&M1	&1/2&	\textbf{0.865}\\
6.7931(17) & {6.7931}&	3/2$^+$ & $^{14}$N+p &	0&	3/2&	{4.75}\\
	&	&	     &  $^{15}$O+$\gamma_{0.00}$	&  E1  &	1/2&	\textbf{2.50$^{\text{a}}$}\\
%	& &       &  $^{15}$O+$\gamma_{6.17}$	&  E1  &	3/2&{-0.002}\\
%6.8594(9) & \textbf{6.8594}&	5/2$^+$&	$^{14}$N+p&	2&	1/2&\textbf{0.39}\\
%	&		&    &     $^{14}$N+p&	2&	3/2&	\textbf{0.42}\\
%	&		&    &     $^{15}$O+$\gamma_{5.24}$&	M1&	5/2&	\textbf{0.04}\\
%7.2759(6) & \textbf{7.2759}&	7/2$^+$&	$^{14}$N+p &	2&	3/2&	\textbf{1541}\\
%	&		&    &     $^{15}$O+$\gamma_{5.24}$&	M1&	5/2&	\textbf{0.00099}\\
\hline
%7.5565(4) & 7.5563	&	1/2$^+$	&	$^{14}$N+p	&	0	&	1/2	&	\textbf{1.0$\times$10$^3$}	\\
%	&	&	&	$^{15}$O+$\gamma_{0.00}$	&	E1	&	1/2	&	\textbf{0.61$\times$10$^{-3}$}\\
%	&	&	&	$^{15}$O+$\gamma_{6.79}$	&	M1	&	3/2	&	\textbf{8.22$\times$10$^{-3}$}\\
%	&	&	&	$^{15}$O+$\gamma_{5.18}$	&	M1	&	1/2	&	\textbf{0.006}\\
%	&	&	&	$^{15}$O+$\gamma_{6.17}$	&	E1	&	3/2	&	\textbf{0.0254}\\
8.2840(5)& \textbf{8.2848}&	3/2$^+$	&	$^{14}$N+p	&	2	&	1/2	&	{-92.2}\\
	&	&	&	$^{14}$N+p	&	0	&	3/2	&	\textbf{4.013$\times$10$^3$}\\
	&	&	&	$^{14}$N+p	&	2	&	3/2	&	{-509}\\
	&	&	&	$^{15}$O+$\gamma_{0.00}$	&	E1	&	1/2	&	\textbf{0.244}\\
%	&	&	&	$^{15}$O+$\gamma_{5.18}$	&	M1	&	1/2	&	{0.01}\\
%	&	&	&	$^{15}$O+$\gamma_{5.24}$	&	M1	&	5/2	&	{0.2}\\
%	&	&	&	$^{15}$O+$\gamma_{6.17}$	&	E1	&	3/2	&	{-4$\times$10$^{-3}$}\\
%	&	&	&	$^{15}$O+$\gamma_{6.86}$	&	M1	&	5/2	&	{0.01}\\
%8.743(6) & \textbf{8.7502}&	1/2$^+$	&	$^{14}$N+p	&	0	&	1/2	&	\textbf{35.726$\times$10$^3$}\\
%	&	&	&	$^{15}$O+$\gamma_{5.18}$	&	M1	&	1/2	&	\textbf{-0.2}\\
%	&	&	&	$^{15}$O+$\gamma_{6.17}$	&	E1	&	3/2	&	\textbf{0.0827}\\
%8.922(2) & \textbf{8.9219}&	5/2$^+$	&	$^{14}$N+p	&	2	&	3/2	&	\textbf{3.8$\times$10$^3$}\\
%	&	&	&	$^{15}$O+$\gamma_{6.79}$	&	M1	&	3/2	&	\textbf{0.003}\\
8.9821(17) & {8.98}&	5/2$^-$	&	$^{14}$N+p	&	1	&	3/2	&	\textbf{-5.872$\times$10$^3$}\\
	&	&	&	$^{15}$O+$\gamma_{0.00}$	&	E2	&	1/2	&	\textbf{-0.303}\\
	&	&	&	$^{15}$O+$\gamma_{6.79}$	&	E1	&	3/2	&	{-0.001}\\
9.484(8) & \textbf{9.488}&	3/2$^+$	&	$^{14}$N+p	&	2	&	1/2	&	{77.69$\times$10$^3$}	\\
	&	&	&	$^{14}$N+p	&	0	&	3/2	&	\textbf{126.685$\times$10$^3$}	\\
	&	&	&	$^{14}$N+p	&	2	&	3/2	&	{-7.822$\times$10$^3$}\\
	&	&	&	$^{15}$O+$\gamma_{0.00}$	&	E1	&	1/2	&	\textbf{6.92}\\
%	&	&	&	$^{15}$O+$\gamma_{6.86}$	&	M1	&	5/2	&	{0.2}\\
9.488(3) & {9.4905}&	5/2$^-$	&	$^{14}$N+p	&	3	&	1/2	&	{0.979$\times$10$^3$}\\
	&	&	&	$^{14}$N+p	&	1	&	3/2	&	{-6.576$\times$10$^3$}\\
	&	&	&	$^{14}$N+p	&	3	&	3/2	&	{-0.985$\times$10$^3$}\\
	&	&	&	$^{15}$O+$\gamma_{0.00}$	&	E2	&	1/2	&	\textbf{-0.307}\\
	&	&	&	$^{15}$O+$\gamma_{6.79}$	&	E1	&	3/2	&	{-0.0123}\\
9.609(2) & {9.6075}&	3/2$^-$	&	$^{14}$N+p	&	1	&	3/2	&	\textbf{-13.821$\times$10$^3$}\\
	&	&	&	$^{15}$O+$\gamma_{0.00}$	&	M1	&	1/2	&	\textbf{1.24}\\
     &	&	&	$^{15}$O+$\gamma_{6.79}$	&	E1	&	3/2	&	{-0.044}\\
%	&	&	&	$^{15}$O+$\gamma_{5.24}$	&	E1	&	5/2	&	{0.095}\\
%10.2817 & \textbf{10.2817}	&	5/2$^+$	&	$^{14}$N+p	&	2	&	3/2	&	\textbf{17.292$\times$10$^3$}\\
%	&	&	&	$^{15}$O+$\gamma_{6.79}$	&	M1	&	3/2	&	\textbf{0.2}\\	
%	&	&	&	$^{15}$O+$\gamma_{6.86}$	&	M1	&	5/2	&	\textbf{-0.4}\\	
%10.480 & \textbf{10.4675}	&	3/2$^-$	&	$^{14}$N+p	&	1	&	1/2	&	\textbf{28.998$\times$10$^3$}\\
%	&	&	&	$^{14}$N+p	&	1	&	3/2	&	\textbf{9.652$\times$10$^3$}\\
%	&	&	&	$^{15}$O+$\gamma_{0.00}$	&	M1	&	1/2	&	\textbf{-0.404}\\	
%	&	&	&	$^{15}$O+$\gamma_{6.79}$	&	E1	&	3/2	&	\textbf{0.1}\\
%	&	&	&	$^{15}$O+$\gamma_{6.86}$	&	E1	&	5/2	&	\textbf{0.1}\\
%10.506 & \textbf{10.5313}&	3/2$^+$	&	$^{14}$N+p	&	0	&	3/2	&	\textbf{205$\times$10$^3$}\\
%	&	&	&	$^{15}$O+$\gamma_{0.00}$	&	E1	&	1/2	&	\textbf{-0.195}\\
%	&	&	&	$^{15}$O+$\gamma_{6.79}$	&	M1	&	3/2	&	\textbf{0.3}\\
%	&	&	&	$^{15}$O+$\gamma_{6.86}$	&	M1	&	5/2	&	\textbf{-0.4}\\
%10.9288 & \textbf{10.9288}&	7/2$^+$	&	$^{14}$N+p	&	2	&	3/2	&	\textbf{56.948$\times$10$^3$}\\
%	&	&	&	$^{15}$O+$\gamma_{6.79}$	&	E2&	3/2	&	\textbf{1}\\
%11.218(3) & 11.217(2)& 3/2$^+$&	$^{14}$N+p	&	0	&	3/2	&	\textbf{40$\times$10$^3$}\\
%	&	&	&	$^{15}$O+$\gamma_{0.00}$	&	E1	&	1/2	& 5.21	\\   	
& {15}	&	3/2$^+$	&	$^{14}$N+p	&	0	&	3/2	&	\textbf{4.722$\times$10$^6$}\\
	&	&	&	$^{15}$O+$\gamma_{0.00}$	&	E1	&	1/2	&	\textbf{327.3}	\\
%& \textbf{15}	&	5/2$^+$	&	$^{14}$N+p	&	2	&	1/2	&	\textbf{1.452$\times$10$^7$}\\
\bottomrule
\end{tabular}
\begin{tablenotes}
\small 
\item Levels used in the \textit{R}-matrix fits. Bold values indicate parameters which were allowed to vary during the fit. The signs on the partial widths and ANCs indicates the relative interferences. The dividing line demarcates the proton separation energy at $E_x$ = 7.2968(5) MeV \cite{Ajzenberg-Selove1991}. Levels where all parameters are fixed are not shown in this table for brevity but were included in the fits. a) Indicates the partial width of the 6.79 MeV state, measured in this experiment to between $\Gamma$ = 0.66 - 3.29 eV. For each individual fit, this width was fixed. However, between each fit, this width was varied to different values within our range to explore how the uncertainty in this measurement affects the low energy extrapolation. These different fits are shown in Fig.\ \ref{fig: rmatrixRange} and are otherwise identical.
\end{tablenotes}
\end{threeparttable}
\label{table: fitParams}
\end{center}
\end{table*}  



The cross-section data utilized in the fitting routine were from measurements at LUNA \cite{Formicola2004, Imbriani2005, Marta2008, Marta2011}, TUNL \cite{Runkle2005}, Bochum \cite{Schroder1987}, and the University of Notre Dame \cite{Li2016}. All of these data sets were left without scaling during the fits. The Bochum data from \citet{Schroder1987} were corrected as detailed in SFII \cite{Adelberger2011}. Additionally, the data used from \citet{Li2016} are a differential cross section taken at 45$\degree$ and are treated as such in the fits. This dataset was scaled by a factor of 4$\pi$ in the plotting only to compare to the angle integrated data. 



\begin{figure}[h!]
\includegraphics[width=1.0\linewidth]{./figures/lifetimeEffects.png}
\caption{$R$-Matrix fits exploring the uncertainty of our lifetime measurements to the low energy extrapolation. The width of the 6.79 MeV excited state in $^{15}$O is fixed during each fit and changed in each subsequent iteration to another value within our uncertainty range. This clearly shows that even though our lifetime result provides the most stringent limitation on the lifetime of this state, it still has an outsized effect on the low energy behavior of this reaction. The Schr{\"{o}}der et al.~data are from \cite{Schroder1987}, while the LUNA data represents the measurements \cite{Formicola2004, Imbriani2005, Marta2008, Marta2011}, the Runkle et al.~data are from \cite{Runkle2005}, and the Li et al.~data are from \cite{Li2016}. Of these, the data used from \citet{Li2016} are differential and were treated as such in the fitting but scaled up by 4$\pi$ for plotting purposes.}
\label{fig: rmatrixRange}
\end{figure}


\begin{figure}[h!]
\includegraphics[width=1.0\linewidth]{./figures/bestFits.png}
\caption{$R$-matrix fits comparing our best fit with the lifetimes to those performed in previous works. Our fit used a lifetime value for the 6.79 MeV excited state in $^{15}$O within our measured range range, $\Gamma=2.75$ where the fits to the data provide good agreement with data above and below the 278 keV resonance. This plot is limited to the low energy region. The Schr{\"{o}}der et al.~data are from \cite{Schroder1987}, while the LUNA data represents the measurements of \cite{Formicola2004, Imbriani2005, Marta2008, Marta2011}, the Runkle et al.~data are from \cite{Runkle2005}, and the Li et al.~data are from \cite{Li2016}. Of these, the data used from \citet{Li2016} are differential and were treated as such in the fitting but scaled up by 4$\pi$ for plotting purposes. The fits from previous works come from Refs.~\cite{Runkle2005, Azuma2010, Adelberger2011, Li2016}.}
\label{fig: rmatrixClose}
\end{figure}

In examining the capture to the ground state in $^{15}$O, the $R$-matrix fits show the effect of our lifetime measurement. Specifically, in Fig.~\ref{fig: rmatrixRange}, we present fits showing the whole range of lifetimes for the 6.79 MeV state of $\tau = 0.6 \pm 0.4$. This shows that despite this measurement providing the most stringent limit on the lifetime, this range still translates into dramatic changes in the low energy behavior of the $S$-factor. Our fits, however, agree well with the capture data and previous studies. One of the best fits using our lifetimes is shown alongside fits from Refs.~\cite{Runkle2005, Azuma2010, Adelberger2011, Li2016} in Fig.~\ref{fig: rmatrixClose}. 

While the experimental results limit the previous uncertainty range in the lifetime data, this $R$-matrix analysis clearly demonstrates that it remains too large for solely reducing uncertainty in the extrapolation for the low energy range cross section of the ground state transition.

\section{Complete fit}
\label{sec: complete fit}


Since our new lifetime measurements still do not significantly limit the extrapolated uncertainty of the low-energy cross section, an $R$-matrix analysis was used to fit all the ground state and the 6.79 MeV transition differential cross section data measured in the current experiment, incorporating the new lifetimes, and an exhaustive set of previously measured data. This, then, provides significantly greater constraint than the lifetime alone and ensures the most robust fit and extrapolation for the low-energy behavior of the $^{14}$N$(p,\gamma)^{15}$O reaction.

Similar to the fits for examining the lifetime's effects, the information about the levels was obtained from  \citet{Ajzenberg-Selove1991} or \citet{Daigle2016} where they had been updated. A channel radius of 5.5 fm was adopted for the fits, which matches the analyses done by Refs.~\cite{Adelberger2011, Li2016, Wagner2018}. Information about the levels and their parameters as used in \texttt{AZURE2} are contained in Table~\ref{table: fitParamsFullFit}. 


\begin{table*}[]
\thisfloatpagestyle{plain}
\caption{PARAMETERS USED IN THE COMPLETE $R$-MATRIX FIT}
\begin{center}
\begin{threeparttable}
\begin{tabular}{c  c  c  c  c  c  c}
\toprule
$E_x$ (Ref.~\cite{Ajzenberg-Selove1991}) &   $E_x$ (fit) & $J^\pi$ & Channel & l & s & ANC (fm$^{-1/2}$) / $\Gamma$ (eV)\\ 
\midrule
0.0 & 0.0	& 1/2$^-$ &	$^{14}$N+p &	1&	1/2&	{0.23}\\
	&	&	    &    $^{14}$N+p &	1&	3/2&	{7.4} \\
%5.183(1) & \textbf{5.183}&	1/2$^+$&	$^{14}$N+p&	0&	1/2&	\textbf{0.33}\\
%	&	&		    &$^{15}$O+$\gamma_{0.00}$  &	E1&	1/2&	\textbf{0.0784}\\
%5.2409(3) & \textbf{5.2409}&	5/2$^+$&	$^{14}$N+p &	2&	1/2&	\textbf{0.23}\\
%                &	&		  & $^{14}$N+p &	2&	3/2&	\textbf{0.24}\\
%	 			&	&	 &$^{15}$O+$\gamma_{0.00}$	&M2&	1/2&	\textbf{0.0002}\\
%6.1763(17) & \textbf{6.1763}&	3/2$^-$& $^{14}$N+p &	1&	1/2&	\textbf{0.47}\\
%				&	&	& $^{14}$N+p	&1	&3/2	&\textbf{0.53}\\
%				&	&	&$^{15}$O+$\gamma_{0.00}$	&M1	&1/2&	\textbf{0.865}\\
6.7931(17) & {6.7931}&	3/2$^+$ & $^{14}$N+p &	0&	3/2&	{4.75}\\
	&	&	     &  $^{15}$O+$\gamma_{0.00}$	&  E1  &	1/2&	\textbf{2.50$^{\text{a}}$}\\
%	& &       &  $^{15}$O+$\gamma_{6.17}$	&  E1  &	3/2&{-0.002}\\
%6.8594(9) & \textbf{6.8594}&	5/2$^+$&	$^{14}$N+p&	2&	1/2&\textbf{0.39}\\
%	&		&    &     $^{14}$N+p&	2&	3/2&	\textbf{0.42}\\
%	&		&    &     $^{15}$O+$\gamma_{5.24}$&	M1&	5/2&	\textbf{0.04}\\
%7.2759(6) & \textbf{7.2759}&	7/2$^+$&	$^{14}$N+p &	2&	3/2&	\textbf{1541}\\
%	&		&    &     $^{15}$O+$\gamma_{5.24}$&	M1&	5/2&	\textbf{0.00099}\\
\hline
%7.5565(4) & 7.5563	&	1/2$^+$	&	$^{14}$N+p	&	0	&	1/2	&	\textbf{1.0$\times$10$^3$}	\\
%	&	&	&	$^{15}$O+$\gamma_{0.00}$	&	E1	&	1/2	&	\textbf{0.61$\times$10$^{-3}$}\\
%	&	&	&	$^{15}$O+$\gamma_{6.79}$	&	M1	&	3/2	&	\textbf{8.22$\times$10$^{-3}$}\\
%	&	&	&	$^{15}$O+$\gamma_{5.18}$	&	M1	&	1/2	&	\textbf{0.006}\\
%	&	&	&	$^{15}$O+$\gamma_{6.17}$	&	E1	&	3/2	&	\textbf{0.0254}\\
8.2840(5)& \textbf{8.2848}&	3/2$^+$	&	$^{14}$N+p	&	2	&	1/2	&	{-92.2}\\
	&	&	&	$^{14}$N+p	&	0	&	3/2	&	\textbf{4.013$\times$10$^3$}\\
	&	&	&	$^{14}$N+p	&	2	&	3/2	&	{-509}\\
	&	&	&	$^{15}$O+$\gamma_{0.00}$	&	E1	&	1/2	&	\textbf{0.244}\\
%	&	&	&	$^{15}$O+$\gamma_{5.18}$	&	M1	&	1/2	&	{0.01}\\
%	&	&	&	$^{15}$O+$\gamma_{5.24}$	&	M1	&	5/2	&	{0.2}\\
%	&	&	&	$^{15}$O+$\gamma_{6.17}$	&	E1	&	3/2	&	{-4$\times$10$^{-3}$}\\
%	&	&	&	$^{15}$O+$\gamma_{6.86}$	&	M1	&	5/2	&	{0.01}\\
%8.743(6) & \textbf{8.7502}&	1/2$^+$	&	$^{14}$N+p	&	0	&	1/2	&	\textbf{35.726$\times$10$^3$}\\
%	&	&	&	$^{15}$O+$\gamma_{5.18}$	&	M1	&	1/2	&	\textbf{-0.2}\\
%	&	&	&	$^{15}$O+$\gamma_{6.17}$	&	E1	&	3/2	&	\textbf{0.0827}\\
%8.922(2) & \textbf{8.9219}&	5/2$^+$	&	$^{14}$N+p	&	2	&	3/2	&	\textbf{3.8$\times$10$^3$}\\
%	&	&	&	$^{15}$O+$\gamma_{6.79}$	&	M1	&	3/2	&	\textbf{0.003}\\
8.9821(17) & {8.98}&	5/2$^-$	&	$^{14}$N+p	&	1	&	3/2	&	\textbf{-5.872$\times$10$^3$}\\
	&	&	&	$^{15}$O+$\gamma_{0.00}$	&	E2	&	1/2	&	\textbf{-0.303}\\
	&	&	&	$^{15}$O+$\gamma_{6.79}$	&	E1	&	3/2	&	{-0.001}\\
9.484(8) & \textbf{9.488}&	3/2$^+$	&	$^{14}$N+p	&	2	&	1/2	&	{77.69$\times$10$^3$}	\\
	&	&	&	$^{14}$N+p	&	0	&	3/2	&	\textbf{126.685$\times$10$^3$}	\\
	&	&	&	$^{14}$N+p	&	2	&	3/2	&	{-7.822$\times$10$^3$}\\
	&	&	&	$^{15}$O+$\gamma_{0.00}$	&	E1	&	1/2	&	\textbf{6.92}\\
%	&	&	&	$^{15}$O+$\gamma_{6.86}$	&	M1	&	5/2	&	{0.2}\\
9.488(3) & {9.4905}&	5/2$^-$	&	$^{14}$N+p	&	3	&	1/2	&	{0.979$\times$10$^3$}\\
	&	&	&	$^{14}$N+p	&	1	&	3/2	&	{-6.576$\times$10$^3$}\\
	&	&	&	$^{14}$N+p	&	3	&	3/2	&	{-0.985$\times$10$^3$}\\
	&	&	&	$^{15}$O+$\gamma_{0.00}$	&	E2	&	1/2	&	\textbf{-0.307}\\
	&	&	&	$^{15}$O+$\gamma_{6.79}$	&	E1	&	3/2	&	{-0.0123}\\
9.609(2) & {9.6075}&	3/2$^-$	&	$^{14}$N+p	&	1	&	3/2	&	\textbf{-13.821$\times$10$^3$}\\
	&	&	&	$^{15}$O+$\gamma_{0.00}$	&	M1	&	1/2	&	\textbf{1.24}\\
     &	&	&	$^{15}$O+$\gamma_{6.79}$	&	E1	&	3/2	&	{-0.044}\\
%	&	&	&	$^{15}$O+$\gamma_{5.24}$	&	E1	&	5/2	&	{0.095}\\
%10.2817 & \textbf{10.2817}	&	5/2$^+$	&	$^{14}$N+p	&	2	&	3/2	&	\textbf{17.292$\times$10$^3$}\\
%	&	&	&	$^{15}$O+$\gamma_{6.79}$	&	M1	&	3/2	&	\textbf{0.2}\\	
%	&	&	&	$^{15}$O+$\gamma_{6.86}$	&	M1	&	5/2	&	\textbf{-0.4}\\	
%10.480 & \textbf{10.4675}	&	3/2$^-$	&	$^{14}$N+p	&	1	&	1/2	&	\textbf{28.998$\times$10$^3$}\\
%	&	&	&	$^{14}$N+p	&	1	&	3/2	&	\textbf{9.652$\times$10$^3$}\\
%	&	&	&	$^{15}$O+$\gamma_{0.00}$	&	M1	&	1/2	&	\textbf{-0.404}\\	
%	&	&	&	$^{15}$O+$\gamma_{6.79}$	&	E1	&	3/2	&	\textbf{0.1}\\
%	&	&	&	$^{15}$O+$\gamma_{6.86}$	&	E1	&	5/2	&	\textbf{0.1}\\
%10.506 & \textbf{10.5313}&	3/2$^+$	&	$^{14}$N+p	&	0	&	3/2	&	\textbf{205$\times$10$^3$}\\
%	&	&	&	$^{15}$O+$\gamma_{0.00}$	&	E1	&	1/2	&	\textbf{-0.195}\\
%	&	&	&	$^{15}$O+$\gamma_{6.79}$	&	M1	&	3/2	&	\textbf{0.3}\\
%	&	&	&	$^{15}$O+$\gamma_{6.86}$	&	M1	&	5/2	&	\textbf{-0.4}\\
%10.9288 & \textbf{10.9288}&	7/2$^+$	&	$^{14}$N+p	&	2	&	3/2	&	\textbf{56.948$\times$10$^3$}\\
%	&	&	&	$^{15}$O+$\gamma_{6.79}$	&	E2&	3/2	&	\textbf{1}\\
%11.218(3) & 11.217(2)& 3/2$^+$&	$^{14}$N+p	&	0	&	3/2	&	\textbf{40$\times$10$^3$}\\
%	&	&	&	$^{15}$O+$\gamma_{0.00}$	&	E1	&	1/2	& 5.21	\\   	
& {15}	&	3/2$^+$	&	$^{14}$N+p	&	0	&	3/2	&	\textbf{4.722$\times$10$^6$}\\
	&	&	&	$^{15}$O+$\gamma_{0.00}$	&	E1	&	1/2	&	\textbf{327.3}	\\
%& \textbf{15}	&	5/2$^+$	&	$^{14}$N+p	&	2	&	1/2	&	\textbf{1.452$\times$10$^7$}\\
\bottomrule
\end{tabular}
\begin{tablenotes}
\small 
\item Levels used in the complete \textit{R}-matrix fit to the whole set of data. Bold values indicate parameters which were allowed to vary during the fit. The signs on the partial widths and ANCs indicates the relative interferences. The dividing line demarcates the proton separation energy at $E_x$ = 7.2968(5) MeV \cite{Ajzenberg-Selove1991}. Levels where all parameters are fixed are not shown in this table for brevity but were included in the fits.
\item \textbf{THIS NEEDS TO BE CHANGED FOR THE NEW FITS WHEN THEY'VE BEEN COMPLETED}
\end{tablenotes}
\end{threeparttable}
\label{table: fitParams}
\end{center}
\end{table*}  



The cross-section data utilized in the fitting routine were from measurements at LUNA \cite{Formicola2004, Imbriani2005, Marta2008, Marta2011}, TUNL \cite{Runkle2005}, Bochum \cite{Schroder1987}, and the University of Notre Dame \cite{Li2016}. All of these data sets were left without scaling during the fits. The Bochum data from \citet{Schroder1987} were corrected as detailed in SFII \cite{Adelberger2011}. Additionally, the data used from \citet{Li2016} are a differential cross section taken at 45$\degree$ and are treated as such in the fits. This dataset was scaled by a factor of 4$\pi$ in the plotting only to compare to the angle integrated data. We chose to exclude the data from \citet{Wagner2018} in the fitting due to its large disagreement to other measurements. This data was, however, shown in plots alongside the other data to show their relation.

%This elastic scattering data incorporated in this analysis were from....

\textbf{FROM HERE, ONCE THE SUMMING CORRECTIONS TO THE GROUND STATE HAVE BEEN COMPLETE I NEED TO RUN THESE FITS AGAIN. THIS WILL THEN LET ME DISCUSS HOW THESE NEW DATA IMPACT THE FITS AND HOW THE EXTRAPOLATION TO LOW ENERGIES BEHAVES. I THINK THAT IS ALL THAT IS LEFT TO FINISH THIS SECTION.}

The extrapolations of the $S$-factors to zero energy for both measured transitions were calculated with the parameters from the \texttt{AZURE2} fit. From these, we recommend values of $S_{6.79}(0) = XXX$ keV b, which is very robust with all fits and agrees well with previous measurements. Furthermore, we recommend $S_{gs}(0) = XXX$ keV b which \textbf{RELATES TO OTHER MEASUREMENTS SOMEHOW}. 


% % uncomment the following lines,
% if using chapter-wise bibliography
%
% \bibliographystyle{ndnatbib}
% \bibliography{example}


%
% Chapter 6
%

%
% Modified by Megan Patnott
% Last Change: Jan 18, 2013
%
%%%%%%%%%%%%%%%%%%%%%%%%%%%%%%%%%%%%%%%%%%%%%%%%%%%%%%%%%%%%%%%%%%%%%%%%
%
% Modified by Sameer Vijay
% Last Change: Tue Jul 26 2005 13:00 CEST
%
%%%%%%%%%%%%%%%%%%%%%%%%%%%%%%%%%%%%%%%%%%%%%%%%%%%%%%%%%%%%%%%%%%%%%%%%
%
% Sample Notre Dame Thesis/Dissertation
% Using Donald Peterson's ndthesis classfile
%
% Written by Jeff Squyres and Don Peterson
%
% Provided by the Information Technology Committee of
%   the Graduate Student Union
%   http://www.gsu.nd.edu/
%
% Nothing in this document is serious except the format.  :-)
%
% If you have any suggestions, comments, questions, please send e-mail
% to: ndthesis@gsu.nd.edu
%
%%%%%%%%%%%%%%%%%%%%%%%%%%%%%%%%%%%%%%%%%%%%%%%%%%%%%%%%%%%%%%%%%%%%%%%%


%
% Chapter 6
%

\chapter{Results and conclusions}
\label{chap: conclusions}



\section{Lifetime measurements}
\label{sec: resultsLifetime}

In this work, we measured the lifetimes of the excited states in $^{15}$O at $E_{x}$ = 5.18 MeV, 6.17 MeV, and 6.79 MeV. The $^{14}$N$(p,\gamma)^{15}$O reaction was used to populate the excited states of $^{15}$O. The nitrogen targets were made by implantation on backings of Mo, Ta, and W in order to examine the effect of different backing materials for the lifetime determination. The Doppler shift of the $\gamma$-rays emitted by the decaying recoils were measured using the different targets and at seven different angles. A Monte Carlo simulation was applied to reproduce the experimental shifts and extract the lifetimes from the measured attenuation factors. By using multiple implanted targets of different backings, we were able to take a weighted average of our measurements to reduce the overall systematic uncertainty. Additionally, the Monte Carlo approach allowed us to recreate the depth profile of implanted targets with a high degree of accuracy, making the subsequent analysis based on the target composition more robust. The simulation also propagates uncertainties throughout every step, allowing it to reflect the experimental conditions more accurately. This is an improvement over previous measurements and their treatment of their targets.


\begin{table}[]
\thisfloatpagestyle{plain}
\caption{MEASURED LIFETIMES AND COMPARISON WITH PREVIOUS MEASUREMENTS}
\begin{center}
\begin{threeparttable}
\begin{tabular}{llllll}
\toprule
$E_{x}$ (keV) & Present       & Ref. \cite{Bertone2001}  & Ref. \cite{Schurmann2008} & Ref. \cite{Galinski2014} & Ref. \cite{Sharma2020} \\
\midrule
5181          & $7.5 \pm 3.0$ & 9.67$^{+1.34}_{-1.24}$ & 8.40$\pm$1.00            & -  & 10.45$^{+2.07}_{-2.21}$                      \\
6172          & $0.7 \pm 0.5$ & 2.10$^{+1.33}_{-1.32}$  & $< 0.77$                 & $< 2.5$   & $< 1.22$              \\
6793          & $0.6 \pm 0.4$ & 1.60$^{+0.75}_{-0.72}$  & $< 0.77$                 & $< 1.8$  & $< 1.18$       \\ \bottomrule
\end{tabular}
\begin{tablenotes}
\small 
\item A summary of the lifetimes (all given in fs) for the excited states in $^{15}$O determined in this work and how they compare with those of previous measurements. 
\end{tablenotes}
\end{threeparttable}
\label{table: lifetimesConclusion}
\end{center}
\end{table}

The results show no evidence of systematic variations with previous measurements arising from the choice of backing materials. This work shows a larger uncertainty for the lifetime of the 5.18 MeV state but agrees within the uncertainties of the previous measurements. For the other transitions at 6.17 MeV and 6.79 MeV, the present measurement agrees well with the values reported in previous works. Our work represents another finite measurement for the lifetime of the 6.79 MeV state, like reported by \citet{Bertone2001}. This work, however, is in agreement with the limits provided by \citet{Schurmann2008}, \citet{Galinski2014}, and \citet{Sharma2020} while providing even more stringent constraints on the level lifetimes. The discrepancies in previous measurements were resolved in this measurement with three different backings. All of our reported lifetimes are given in comparison with those of previous measurements in Table \ref{table: lifetimesConclusion}.


\section{Cross section results}
\label{sec: csResults}

The excitation function of the $^{14}$N$(p,\gamma)^{15}$O reaction has been measured with a HPGe detector from $E_{p}$ = 0.27 to 1.07~MeV for the ground state and 6.79~MeV transitions at the CASPAR facility. These measurements bridge the gap between low-energy measurements of this reaction \cite{Formicola2004, Imbriani2005, Marta2008, Marta2011, Runkle2005} and those at high energy \cite{Schroder1987, Li2016}. The differential cross section and $S$-factor from these new measurements were determined. It was also determined that it was most appropriate to treat the corrected data from \cite{Schroder1987} as differential when used in the $R$-matrix fits. Ultimately, the present results agreed well with measurements of \cite{Schroder1987, Imbriani2005, Runkle2005, Marta2011, Li2016, Wagner2018} across the energy range in question for both transitions, with the notable exception that the same enhancement in the 6.79 MeV transition seen in \citet{Wagner2018} could not be confirmed here. 

While these measurements carried higher uncertainties than prior data, this is due primarily to time constraints and they could all be reduced to comparable levels by taking more data. This work demonstrates the effectiveness of the CASPAR facility and its capacity for further, astrophysically relevant measurements. 

A multichannel $R$-matrix analysis was performed simultaneously for both the ground state and $E_x$ = 6.79~MeV transitions. Incorporating recent results for the lifetime of the excited state at 6.79~MeV \cite{Frentz2021}, we find the extrapolated resulting zero-energy $S$-factor components for each of the two transitions are $S_{g.s.}(0) = 0.33_{-0.08}^{+0.16}$ keV b and $S_{6.79}(0) = 1.24 \pm 0.09 $ keV b. These reported uncertainties reflect the fact that there are clear, systematic differences between the measured low-energy data of \citet{Imbriani2005} and \citet{Runkle2005} that are not being effectively captured in the $R$-matrix fit. These uncertainties were thus chosen to align our results to those prior measurements. 

The value for the $S_{6.79}(0)$ overlaps exactly with that of \citet{Wagner2018} and in between the reported values of \citet{Li2016} and the previously accepted value from \citet{Adelberger2011}. Overall, this indicates that this transition's extrapolation is quite robust to the addition of new data, even with some discrepancies like that coming from the elevated cross sections at the high-energy end of \citet{Wagner2018}. For the ground state zero-energy extrapolation, $S_{g.s.}(0)$, is higher than that presented in either \citet{Wagner2018} or \citet{Adelberger2011}, but still lower than from \citet{Li2016}. 

The approach taken here for the study of the $^{14}$N$(p,\gamma)^{15}$O reaction has succeeded in combining both the efforts of an improved lifetime measurement and new cross section data in underground accelerator experiments. The combination of these two complementary measurements allows us to suggest an enhancement to the low-energy behavior of the $^{14}$N$(p,\gamma)^{15}$O reaction, complementing the higher rate observed in the recent Borexino Solar CNO neutrino measurements \cite{agostini2020direct}. 


\section{$^{14}$N$\left( p,\gamma \right) ^{15}$O reaction outlook}
\label{sec: outlook}

At this point, there are still uncertainties concerning the low-energy behavior of the $^{14}$N$\left( p,\gamma \right) ^{15}$O reaction. Between the results shown in this work, providing the most stringent constraint of the lifetime of the 6.79~MeV state~\cite{Frentz2021} and an updated evaluation of the cross-section, $S$-factor, and reaction rate, it can be broadly concluded that the largest sources of uncertainty within this reaction now lies in the weaker transitions, specifically at low energies. Additional measurements of the ground-state transition at low energies could yield further insights, particularly with angular distribution measurements at energies below those performed by \citet{Li2016} and additional measurements below the 278~keV resonance.

A longer measurement of the capture to ground state at energies lower than the \citet{Imbriani2005} measurement would be the best method to reduce the uncertainties. Beyond that, the region immediately above the $E_{p}$ = 278 keV resonance in the $^{14}$N$\left( p,\gamma \right) ^{15}$O reaction is immensely difficult due to the diffusion of nitrogen into target backings. With how strong the resonance is when compared to the direct capture for the ground state, any amount of nitrogen that diffuses into the backing will cause an elevation in the data at energies above the resonance. Longer term measurements with more systematic studies in that immediate region specifically, from approximately $E_{p}$ = 300 - 400 keV, could provide a more concrete understanding of the resonance at 278 keV and the behavior of the reaction below. Such investigations would be ideally suited for one of the CASPAR~\cite{Robertson2016}, LUNA \cite{FORMICOLA2003609} or the newly installed JUNA facility~\cite{liu2016}.

In terms of doing a better measurement of the important lifetimes, our work shows that any improvement to be gained would require more data. This work has shown that taking more measurements with even more varying targets would allow for an even more precise measurements, especially when taken in aggregate as a weighted average. A measurement with a state-of-the-art detection array with gamma tracking, like GRETINA at FRIB, could potentially provide enough angular precision to give further constraint on these extremely low lifetimes. However, as was seen, it would take a drastic reduction in the lifetime uncertainty to provide a meaningful constraint on the low-energy reaction cross-section. Such a reduction is likely to be impossible without state-of-the-art facilities due to the lifetime's proximity to the DSAM technique's limit. Our method of combining measurements of varied backing materials with a weighted average, however, is a powerful approach and has pushed conventional DSAM techniques to their lowest limit, providing further clarity about the nature of the $^{14}$N$\left( p,\gamma \right) ^{15}$O reaction. 



% % uncomment the following lines,
% if using chapter-wise bibliography
%
% \bibliographystyle{ndnatbib}
% \bibliography{example}


%
% Chapter 7
%

%\include{chapter7}


%
% Appendix (optional)
%

\appendix

%
% Modified by Sameer Vijay
% Last Change: Wed Jul 27 2005 13:00 CEST
%
%%%%%%%%%%%%%%%%%%%%%%%%%%%%%%%%%%%%%%%%%%%%%%%%%%%%%%%%%%%%%%%%%%%%%%%%
%
% Sample Notre Dame Thesis/Dissertation
% Using Donald Peterson's ndthesis classfile
%
% Written by Jeff Squyres and Don Peterson
%
% Provided by the Information Technology Committee of
%   the Graduate Student Union
%   http://www.gsu.nd.edu/
%
% Nothing in this document is serious except the format.  :-)
%
% If you have any suggestions, comments, questions, please send e-mail
% to: ndthesis@gsu.nd.edu
%
%%%%%%%%%%%%%%%%%%%%%%%%%%%%%%%%%%%%%%%%%%%%%%%%%%%%%%%%%%%%%%%%%%%%%%%%

%%%%%%%%%%%%%%%%%%%%%%%%%%%%%%%%%%%%%%%%%%%%%%%%%%%%%%%%%%%%%%%%%%%%%%%%
%
% Appendix
%
%%%%%%%%%%%%%%%%%%%%%%%%%%%%%%%%%%%%%%%%%%%%%%%%%%%%%%%%%%%%%%%%%%%%%%%%

\chapter{Analysis codes}
\label{appendix: codes}

\section{Definitions}

Several codes used within this analysis are reproduced herein for posterity and ease of access.

\lstinputlisting[language=C]{./codes/calibrate.cpp}

% % uncomment the following lines,
% if using chapter-wise bibliography
%
% \bibliographystyle{ndnatbib}
% \bibliography{example}


%
% Modified by Sameer Vijay
% Last Change: Wed Jul 27 2005 13:00 CEST
%
%%%%%%%%%%%%%%%%%%%%%%%%%%%%%%%%%%%%%%%%%%%%%%%%%%%%%%%%%%%%%%%%%%%%%%%%
%
% Sample Notre Dame Thesis/Dissertation
% Using Donald Peterson's ndthesis classfile
%
% Written by Jeff Squyres and Don Peterson
%
% Provided by the Information Technology Committee of
%   the Graduate Student Union
%   http://www.gsu.nd.edu/
%
% Nothing in this document is serious except the format.  :-)
%
% If you have any suggestions, comments, questions, please send e-mail
% to: ndthesis@gsu.nd.edu
%
%%%%%%%%%%%%%%%%%%%%%%%%%%%%%%%%%%%%%%%%%%%%%%%%%%%%%%%%%%%%%%%%%%%%%%%%

%%%%%%%%%%%%%%%%%%%%%%%%%%%%%%%%%%%%%%%%%%%%%%%%%%%%%%%%%%%%%%%%%%%%%%%%
%
% Appendix
%
%%%%%%%%%%%%%%%%%%%%%%%%%%%%%%%%%%%%%%%%%%%%%%%%%%%%%%%%%%%%%%%%%%%%%%%%

\chapter{Publications}
\label{appendix: pubs}

Included in this appendix are the publications that have come from this work.



\section{Lifetime measurements of excited states in $^{15}$O}

\textbf{GIVE THE FULL CITATION FOR THE PAPER HERE WHEN I KNOW IT HAS BEEN ACCEPTED.}

\includepdf[pages=-,pagecommand={},width=\textwidth]{./publications/lifetimesPaper.pdf}

% % uncomment the following lines,
% if using chapter-wise bibliography
%
% \bibliographystyle{ndnatbib}
% \bibliography{example}



%
% Back stuff
%

 \backmatter
 \bibliographystyle{nddiss2e} % The standard abbrvnat style should be acceptable. Also provided with both the advanced and standard
% \bibliography{example}       % distributions are nddiss2e and nddiss2enoarticletitles style options.
 \bibliography{dissertationBibliography}


\end{document}

% End of ``example.tex''
