%
% Modified by Megan Patnott
% Last Change: Jan 18, 2013
%
%%%%%%%%%%%%%%%%%%%%%%%%%%%%%%%%%%%%%%%%%%%%%%%%%%%%%%%%%%%%%%%%%%%%%%%%
%
% Modified by Bryce Frentz
% Last Change: 2018
%
%%%%%%%%%%%%%%%%%%%%%%%%%%%%%%%%%%%%%%%%%%%%%%%%%%%%%%%%%%%%%%%%%%%%%%%%
%
% Sample Notre Dame Thesis/Dissertation
% Using Donald Peterson's ndthesis classfile
%
% Written by Jeff Squyres and Don Peterson
%
% Provided by the Information Technology Committee of
%   the Graduate Student Union
%   http://www.gsu.nd.edu/
%
% Nothing in this document is serious except the format.  :-)
%
% If you have any suggestions, comments, questions, please send e-mail
% to: ndthesis@gsu.nd.edu
%
%%%%%%%%%%%%%%%%%%%%%%%%%%%%%%%%%%%%%%%%%%%%%%%%%%%%%%%%%%%%%%%%%%%%%%%%

%
% Chapter 2
%

\chapter{Experimental setup and procedures}
\label{chap: experiment}

\section{Introduction}

This chapter is included to highlight some techniques and equipment commonly used in low-energy nuclear physics, with an emphasis on those used in this work. The discussion will be presented from the lens of the general to the specific, introducing general concepts for beam production and transport, acceleration systems, and radiation detection. Following the discussion of the scientific principles, I will detail the specific and the specifics of the equipment employed in this collective work. 


\section{Accelerators, beam production, and radiation detection}
\label{sec: accelerators, beamline, and detectors}



\subsection{Van de Graaff accelerators}
\label{sec: vdg accelerators}

First and foremost, low-energy nuclear physics (generally understood to be the regime in which incident kinetic energies for reactions are below 1 GeV, often even as low as sub-MeV) requires particle acceleration systems in order to bring beams and targets together. The most common techniques for accelerating ions are 1) to use strong electrostatic fields for a straight-line acceleration, such as in the Van de Graaff accelerator, or 2)  by using a combination of electric and magnetic fields to accelerate the particles in cyclotron motion, aptly named a cyclotron. Van de Graaff accelerators are the most prominent of all accelerators, though, and are the type used in this work. 

The basic idea of a Van de Graaff accelerator is that a beam of ions produced in an ion source (Section \ref{sec: ion sources}) are manipulated by additional electromagnetic elements (Section \ref{sec: beamline}) into an area of high electric potential. This high voltage is produced and maintained by continuously transporting positive charge from ground to a Faraday cage with a field free interior (referred to as the terminal, with voltage $V_{T}$). Upon exiting the accelerator, the ions have energy

\begin{equation}
E_{ion} = qV_{T} = \dfrac{1}{2} m_{ion} v_{ion}^{2}
\label{eqn: accelerator}
\end{equation}

\noindent where $V_{T}$ again is the terminal voltage and $q$ is the charge state of the ion going through the accelerator, $m_{ion}$ is the mass of the ion, and $v_{ion}$ is the ion's velocity. When the terminal voltage is in MV, the beam energy is then provided in MeV. 

The charge is delivered to the terminal via a mechanical delivery system. The 5U Sta. Ana accelerator at the NSL uses four Pelletron chains cite{Herb1974}, each with its own power supply 


\subsection{Ion sources}
\label{sec: ion sources}



\subsection{Beam transport}
\label{sec: beamline}



\subsection{High-purity Germanium detectors}
\label{sec: HPGe detectors}


\section{Cross-section measurements}
\label{sec: cs experiment}



This set of data was taken over the course of five* separate experiments. The first occurred at the University of Notre Dame's Nuclear Science Laboratory (NSL) in January of 2018 and covered the proton energy range of E$_{p}$ = 800 - 1200 keV. The experiment was then continued at the Compact Accelerator System for Performing Nuclear Astrophysics (CASPAR) facility at the Sanford Underground Research Facility located in Lead, South Dakota in three increments: February 2018, May 2018, and August / September 2018. These measurements covered the energy range from E$_{p}$ = 270 - 1200 keV, in order to measure the $^{14}$N$\left( p,\gamma \right) ^{15}$O reaction cross-section to compare the performance of the CASPAR facility to an above-ground laboratory. Finally, in MONTH TIME DATE THING, the final experiment was completed at the NSL, focusing on obtaining the lifetime of the 6793 keV state in $^{15}$O.

\subsection{Measurement at Notre Dame}

\subsection{Measurment at CASPAR}


\section{Lifetime measurement}
\label{sec: lifetime experiment}

\subsection{Target production}

\subsection{Measurement at Notre Dame}




% % uncomment the following lines,
% if using chapter-wise bibliography
%
% \bibliographystyle{ndnatbib}
% \bibliography{example}
