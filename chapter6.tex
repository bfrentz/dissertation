%
% Modified by Megan Patnott
% Last Change: Jan 18, 2013
%
%%%%%%%%%%%%%%%%%%%%%%%%%%%%%%%%%%%%%%%%%%%%%%%%%%%%%%%%%%%%%%%%%%%%%%%%
%
% Modified by Sameer Vijay
% Last Change: Tue Jul 26 2005 13:00 CEST
%
%%%%%%%%%%%%%%%%%%%%%%%%%%%%%%%%%%%%%%%%%%%%%%%%%%%%%%%%%%%%%%%%%%%%%%%%
%
% Sample Notre Dame Thesis/Dissertation
% Using Donald Peterson's ndthesis classfile
%
% Written by Jeff Squyres and Don Peterson
%
% Provided by the Information Technology Committee of
%   the Graduate Student Union
%   http://www.gsu.nd.edu/
%
% Nothing in this document is serious except the format.  :-)
%
% If you have any suggestions, comments, questions, please send e-mail
% to: ndthesis@gsu.nd.edu
%
%%%%%%%%%%%%%%%%%%%%%%%%%%%%%%%%%%%%%%%%%%%%%%%%%%%%%%%%%%%%%%%%%%%%%%%%


%
% Chapter 6
%

\chapter{Results and conclusions}
\label{chap: conclusions}



\section{Lifetime measurements}
\label{sec: resultsLifetime}

In this work, we measured the lifetimes of the excited states in $^{15}$O at $E_{x}$ = 5.18 MeV, 6.17 MeV, and 6.79 MeV. The $^{14}$N$(p,\gamma)^{15}$O reaction was used to populate the excited states of $^{15}$O. The nitrogen targets were made by implantation on backings of Mo, Ta, and W in order to examine the effect of different backing materials for the lifetime determination. The Doppler shift of the $\gamma$-rays emitted by the decaying recoils were measured using the different targets and at seven different angles. A Monte Carlo simulation was applied to reproduce the experimental shifts and extract the lifetimes from the measured attenuation factors. By using multiple implanted targets of different backings, we were able to take a weighted average of our measurements to reduce the overall systematic uncertainty. Additionally, the Monte Carlo approach allowed us to recreate the depth profile of implanted targets with a high degree of accuracy, making the subsequent analysis based on the target composition more robust. The simulation also propagates uncertainties throughout every step, allowing it to reflect the experimental conditions more accurately. This is an improvement over previous measurements and their treatment of their targets.


\begin{table}[]
\thisfloatpagestyle{plain}
\caption{MEASURED LIFETIMES AND COMPARISON WITH PREVIOUS MEASUREMENTS}
\begin{center}
\begin{threeparttable}
\begin{tabular}{llllll}
\toprule
$E_{x}$ (keV) & Present       & Ref. \cite{Bertone2001}  & Ref. \cite{Schurmann2008} & Ref. \cite{Galinski2014} & Ref. \cite{Sharma2020} \\
\midrule
5181          & $7.5 \pm 3.0$ & 9.67$^{+1.34}_{-1.24}$ & 8.40$\pm$1.00            & -  & 10.45$^{+2.07}_{-2.21}$                      \\
6172          & $0.7 \pm 0.5$ & 2.10$^{+1.33}_{-1.32}$  & $< 0.77$                 & $< 2.5$   & $< 1.22$              \\
6793          & $0.6 \pm 0.4$ & 1.60$^{+0.75}_{-0.72}$  & $< 0.77$                 & $< 1.8$  & $< 1.18$       \\ \bottomrule
\end{tabular}
\begin{tablenotes}
\small 
\item A summary of the lifetimes (all given in fs) for the excited states in $^{15}$O determined in this work and how they compare with those of previous measurements. 
\end{tablenotes}
\end{threeparttable}
\label{table: lifetimesConclusion}
\end{center}
\end{table}

The results show no evidence of systematic variations with previous measurements arising from the choice of backing materials. This work shows a larger uncertainty for the lifetime of the 5.18 MeV state but agrees within the uncertainties of the previous measurements. For the other transitions at 6.17 MeV and 6.79 MeV, the present measurement agrees well with the values reported in previous works. Our work represents another finite measurement for the lifetime of the 6.79 MeV state, like reported by \citet{Bertone2001}. This work, however, is in agreement with the limits provided by \citet{Schurmann2008}, \citet{Galinski2014}, and \citet{Sharma2020} while providing even more stringent constraints on the level lifetimes. The discrepancies in previous measurements were resolved in this measurement with three different backings. All of our reported lifetimes are given in comparison with those of previous measurements in Table \ref{table: lifetimesConclusion}.


\section{Cross section results}
\label{sec: csResults}

The excitation function of the $^{14}$N$(p,\gamma)^{15}$O reaction has been measured with a HPGe detector from $E_{p}$ = 0.27 to 1.07~MeV for the ground state and 6.79~MeV transitions at the CASPAR facility. These measurements bridge the gap between low-energy measurements of this reaction \cite{Formicola2004, Imbriani2005, Marta2008, Marta2011, Runkle2005} and those at high energy \cite{Schroder1987, Li2016}. The differential cross section and $S$-factor from these new measurements were determined. It was also determined that it was most appropriate to treat the corrected data from \cite{Schroder1987} as differential when used in the $R$-matrix fits. Ultimately, the present results agreed well with measurements of \cite{Schroder1987, Imbriani2005, Runkle2005, Marta2011, Li2016, Wagner2018} across the energy range in question for both transitions, with the notable exception that the same enhancement in the 6.79 MeV transition seen in \citet{Wagner2018} could not be confirmed here. 

While these measurements carried higher uncertainties than prior data, this is due primarily to time constraints and they could all be reduced to comparable levels by taking more data. This work demonstrates the effectiveness of the CASPAR facility and its capacity for further, astrophysically relevant measurements. 

A multichannel $R$-matrix analysis was performed simultaneously for both the ground state and $E_x$ = 6.79~MeV transitions. Incorporating recent results for the lifetime of the excited state at 6.79~MeV \cite{Frentz2021}, we find the extrapolated resulting zero-energy $S$-factor components for each of the two transitions are $S_{g.s.}(0) = 0.33_{-0.08}^{+0.16}$ keV b and $S_{6.79}(0) = 1.24 \pm 0.09 $ keV b. These reported uncertainties reflect the fact that there are clear, systematic differences between the measured low-energy data of \citet{Imbriani2005} and \citet{Runkle2005} that are not being effectively captured in the $R$-matrix fit. These uncertainties were thus chosen to align our results to those prior measurements. 

The value for the $S_{6.79}(0)$ overlaps exactly with that of \citet{Wagner2018} and in between the reported values of \citet{Li2016} and the previously accepted value from \citet{Adelberger2011}. Overall, this indicates that this transition's extrapolation is quite robust to the addition of new data, even with some discrepancies like that coming from the elevated cross sections at the high-energy end of \citet{Wagner2018}. For the ground state zero-energy extrapolation, $S_{g.s.}(0)$, is higher than that presented in either \citet{Wagner2018} or \citet{Adelberger2011}, but still lower than from \citet{Li2016}. 

The approach taken here for the study of the $^{14}$N$(p,\gamma)^{15}$O reaction has succeeded in combining both the efforts of an improved lifetime measurement and new cross section data in underground accelerator experiments. The combination of these two complementary measurements allows us to suggest an enhancement to the low-energy behavior of the $^{14}$N$(p,\gamma)^{15}$O reaction, complementing the higher rate observed in the recent Borexino Solar CNO neutrino measurements \cite{agostini2020direct}. 


\section{$^{14}$N$\left( p,\gamma \right) ^{15}$O reaction outlook}
\label{sec: outlook}

At this point, there are still uncertainties concerning the low-energy behavior of the $^{14}$N$\left( p,\gamma \right) ^{15}$O reaction. Between the results shown in this work, providing the most stringent constraint of the lifetime of the 6.79~MeV state~\cite{Frentz2021} and an updated evaluation of the cross-section, $S$-factor, and reaction rate, it can be broadly concluded that the largest sources of uncertainty within this reaction now lies in the weaker transitions, specifically at low energies. Additional measurements of the ground-state transition at low energies could yield further insights, particularly with angular distribution measurements at energies below those performed by \citet{Li2016} and additional measurements below the 278~keV resonance.

A longer measurement of the capture to ground state at energies lower than the \citet{Imbriani2005} measurement would be the best method to reduce the uncertainties. Beyond that, the region immediately above the $E_{p}$ = 278 keV resonance in the $^{14}$N$\left( p,\gamma \right) ^{15}$O reaction is immensely difficult due to the diffusion of nitrogen into target backings. With how strong the resonance is when compared to the direct capture for the ground state, any amount of nitrogen that diffuses into the backing will cause an elevation in the data at energies above the resonance. Longer term measurements with more systematic studies in that immediate region specifically, from approximately $E_{p}$ = 300 - 400 keV, could provide a more concrete understanding of the resonance at 278 keV and the behavior of the reaction below. Such investigations would be ideally suited for one of the CASPAR~\cite{Robertson2016}, LUNA \cite{FORMICOLA2003609} or the newly installed JUNA facility~\cite{liu2016}.

In terms of doing a better measurement of the important lifetimes, our work shows that any improvement to be gained would require more data. This work has shown that taking more measurements with even more varying targets would allow for an even more precise measurements, especially when taken in aggregate as a weighted average. A measurement with a state-of-the-art detection array with gamma tracking, like GRETINA at FRIB, could potentially provide enough angular precision to give further constraint on these extremely low lifetimes. However, as was seen, it would take a drastic reduction in the lifetime uncertainty to provide a meaningful constraint on the low-energy reaction cross-section. Such a reduction is likely to be impossible without state-of-the-art facilities due to the lifetime's proximity to the DSAM technique's limit. Our method of combining measurements of varied backing materials with a weighted average, however, is a powerful approach and has pushed conventional DSAM techniques to their lowest limit, providing further clarity about the nature of the $^{14}$N$\left( p,\gamma \right) ^{15}$O reaction. 



% % uncomment the following lines,
% if using chapter-wise bibliography
%
% \bibliographystyle{ndnatbib}
% \bibliography{example}
