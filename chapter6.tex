%
% Modified by Megan Patnott
% Last Change: Jan 18, 2013
%
%%%%%%%%%%%%%%%%%%%%%%%%%%%%%%%%%%%%%%%%%%%%%%%%%%%%%%%%%%%%%%%%%%%%%%%%
%
% Modified by Sameer Vijay
% Last Change: Tue Jul 26 2005 13:00 CEST
%
%%%%%%%%%%%%%%%%%%%%%%%%%%%%%%%%%%%%%%%%%%%%%%%%%%%%%%%%%%%%%%%%%%%%%%%%
%
% Sample Notre Dame Thesis/Dissertation
% Using Donald Peterson's ndthesis classfile
%
% Written by Jeff Squyres and Don Peterson
%
% Provided by the Information Technology Committee of
%   the Graduate Student Union
%   http://www.gsu.nd.edu/
%
% Nothing in this document is serious except the format.  :-)
%
% If you have any suggestions, comments, questions, please send e-mail
% to: ndthesis@gsu.nd.edu
%
%%%%%%%%%%%%%%%%%%%%%%%%%%%%%%%%%%%%%%%%%%%%%%%%%%%%%%%%%%%%%%%%%%%%%%%%


%
% Chapter 6
%

\chapter{Results and conclusions}
\label{chap: conclusions}



\section{Lifetime measurements}
\label{sec: resultsLifetime}

In this work, we measured the lifetimes of the excited states in $^{15}$O at $E_{x}$ = 5.18 MeV, 6.17 MeV, and 6.79 MeV. The $^{14}$N$(p,\gamma)^{15}$O reaction was used to populate the excited states of $^{15}$O. The nitrogen targets were made by implantation on backings of Mo, Ta, and W in order to examine the effect of different backing materials for the lifetime determination. The Doppler shift of the $\gamma$-rays emitted by the decaying recoils were measured using the different targets and at seven different angles. A Monte Carlo simulation was applied to reproduce the experimental shifts and extract the lifetimes from the measured attenuation factors. By using multiple implanted targets of different backings, we were able to take a weighted average of our measurements to reduce the overall systematic uncertainty. Additionally, the Monte Carlo approach allowed us to recreate the depth profile of implanted targets with a high degree of accuracy, making the subsequent analysis based on the target composition more robust. The simulation also propagates uncertainties throughout every step, allowing it to reflect the experimental conditions more accurately. This is an improvement over previous measurements and their treatment of their targets.


\begin{table}[]
\thisfloatpagestyle{plain}
\caption{MEASURED LIFETIMES AND COMPARISON WITH PREVIOUS MEASUREMENTS}
\begin{center}
\begin{threeparttable}
\begin{tabular}{llllll}
\toprule
$E_{x}$ (keV) & Present       & Ref. \cite{Bertone2001}  & Ref. \cite{Schurmann2008} & Ref. \cite{Galinski2014} & Ref. \cite{Sharma2020} \\
\midrule
5181          & $7.5 \pm 3.0$ & 9.67$^{+1.34}_{-1.24}$ & 8.40$\pm$1.00            & -  & 10.45$^{+2.07}_{-2.21}$                      \\
6172          & $0.7 \pm 0.5$ & 2.10$^{+1.33}_{-1.32}$  & $< 0.77$                 & $< 2.5$   & $< 1.22$              \\
6793          & $0.6 \pm 0.4$ & 1.60$^{+0.75}_{-0.72}$  & $< 0.77$                 & $< 1.8$  & $< 1.18$       \\ \bottomrule
\end{tabular}
\begin{tablenotes}
\small 
\item A summary of the lifetimes (all given in fs) for the excited states in $^{15}$O determined in this work and how they compare with those of previous measurements. 
\end{tablenotes}
\end{threeparttable}
\label{table: lifetimesConclusion}
\end{center}
\end{table}

The results show no evidence of systematic variations with previous measurements arising from the choice of backing materials. This work shows a larger uncertainty for the lifetime of the 5.18 MeV state but agrees within the uncertainties of the previous measurements. For the other transitions at 6.17 MeV and 6.79 MeV, the present measurement agrees well with the values reported in previous works. Our work represents another finite measurement for the lifetime of the 6.79 MeV state, like reported by \citet{Bertone2001}. This work, however, is in agreement with the limits provided by \citet{Schurmann2008}, \citet{Galinski2014}, and \citet{Sharma2020} while providing even more stringent constraints on the level lifetimes. The discrepancies in previous measurements were resolved in this measurement with three different backings. All of our reported lifetimes are given in comparison with those of previous measurements in Table \ref{table: lifetimesConclusion}.


\section{Cross section results}
\label{sec: csResults}

In this work, the ground state transitions and the R/DC $\rightarrow$ 6.79 MeV transition from the $^{14}$N$(p,\gamma)^{15}$O reaction have been measured at the underground CASPAR facility. The reaction was conducted at energies between 270 keV and 1200 keV in order to bridge the gap between previously disconnected sets of measurements. Our data exhibits excellent agreement to the previously published measurements \cite{Formicola2004, Imbriani2005, Runkle2005, Marta2008, Marta2011, Li2016} and reasonable agreement with the work of \cite{Schroder1987, Wagner2018}. While our measurements carried higher uncertainties than prior data, this is due primarily to time constraints and they could all be reduced to comparable levels by taking more data. This work demonstrates the effectiveness of the CASPAR facility and its capacity for further, astrophysically relevant measurements. 

From our extrapolation, we recommend $S_{6.79}(0) = XXX$ keV b, which is very robust with all fits and agrees well with previous measurements. Furthermore, we recommend $S_{gs}(0) = XXX$ keV b which \textbf{RELATES TO OTHER MEASUREMENTS SOMEHOW}. 

The results are summarized in Table \ref{table: sfactorSummary} together with literature values.


\begin{sidewaystable*}[]
\thisfloatpagestyle{plain}
\caption{SUMMARY OF THE EXTRAPOLATED $S$-FACTORS}
\begin{center}
\begin{threeparttable}
\begin{tabular}{llll}
\toprule
Year & Reference & $S_{gs}(0)$       & $S_{6.79}(0)$ \\
\midrule
1987 & \citet{Schroder1987}   & $0.25 \pm 0.06$ & $1.20 \pm 0.05$  \\
2001 & \citet{Angulo2001}  (a) & 0.08$^{+0.13}_{-0.06}$  & 1.63 $\pm$ 0.17 \\
2003 & \citet{Mukhamedzhanov2003} & 0.15 $\pm$ 0.07 & 1.40 $\pm$ 0.20 \\
2004 & \citet{Formicola2004}  & 0.25 $\pm$ 0.06 & 1.35 $\pm$ 0.05 (stat)$\pm$ 0.08 (sys) \\
2005 &\citet{Angulo2005}   & 0.25 $\pm$ 0.08  & 1.35 $\pm$ 0.04  \\
2005 & \citet{Imbriani2005}   & $0.25 \pm 0.06$ & $1.20 \pm 0.05$  \\
2005 & \citet{Runkle2005}   & $0.49 \pm 0.08$ & $1.15 \pm 0.05$  \\
2008 & \citet{Marta2008}  & 0.20 $\pm$ 0.05   & $--$ \\
2010 & \citet{Azuma2010}   & $0.28$ & $1.30$  \\
2011 & \citet{Adelberger2011}   & $0.27 \pm 0.05$ & $1.18 \pm 0.05$  \\
2016 & \citet{Li2016}   & 0.42 $\pm$ 0.04 (stat)  $^{+0.09}_{-0.19}$(sys) & 1.29 $\pm$ 0.06 (stat) $\pm$ 0.06 (sys)  \\
2018 & \citet{Wagner2018}   & 0.19 $\pm$ 0.01 (stat) $\pm$ 0.05 (sys) & 1.24 $\pm$ 0.02 (stat) $\pm$ 0.11 (sys)  \\ 
2021 & Present work  & $XXX5 \pm XXX$ & $XXX \pm XXX$  \\
\bottomrule
\end{tabular}
\begin{tablenotes}
\small 
\item A summary of the zero-energy $S$-factors (all given in keV b) for both the ground state and 6.79 MeV excited state transition in $^{15}$O. Shows this work's results and how they compare with those of previous measurements.  (a) $R$-matrix analysis of available data, not a measurement.
\end{tablenotes}
\end{threeparttable}
\label{table: sfactorSummary}
\end{center}
\end{sidewaystable*}



The approach taken here for the study of the $^{14}$N$(p,\gamma)^{15}$O reaction has succeeded in combining both the efforts of an improved lifetime measurement and new cross section data in underground accelerator experiments. The combination of these two complementary measurements allows us to provide a consistent picture of this reaction and its behavior. 


\section{$^{14}$N$\left( p,\gamma \right) ^{15}$O reaction outlook}
\label{sec: outlook}

At this point, there are still uncertainties concerning the low-energy behavior of the $^{14}$N$\left( p,\gamma \right) ^{15}$O reaction. A longer measurement of the capture to ground state at energies lower than the LUNA measurement would be the best method to reduce the uncertainties. Beyond that, much comes down to how the data is interpreted. The region immediately above the $E_{p}$ = 278 keV resonance in the $^{14}$N$\left( p,\gamma \right) ^{15}$O reaction is immensely difficult due to the diffusion of nitrogen into target backings. With how strong the resonance is when compared to the direct capture for the ground state, any amount of nitrogen that diffuses into the backing will cause an elevation in the data at energies above the resonance. Therefore, if the data is cut out in the region of $\sim 300 - 400$ keV and ignored, the fit might be more accurate and provide a higher quality understanding of the behavior at low energies. For further improvements to the understanding of the reaction from cross-section measurements, taking more data at low energies coinciding with the LUNA measurements or perhaps exploring even lower reaction energies, which might be possible at the new LUNA-MV or JUNA facilities, would provide confirmation and constraint to strengthen the understanding of the reaction. 

In terms of doing a better measurement of the important lifetimes, our work shows that any improvement to be gained would require more data. This work has shown that taking more measurements with even more varying targets would allow for an even more precise measurements, especially when taken in aggregate as a weighted average. A measurement with a state-of-the-art detection array with gamma tracking, like GRETINA at FRIB, could potentially provide enough angular precision to give further constraint on these extremely low lifetimes. However, as was seen, it would take a drastic reduction in the lifetime uncertainty alone to provide a meaningful constraint on the low-energy reaction cross-section. Such a reduction is likely to be impossible due to the lifetime's proximity to the DSAM technique's limit. Our method of combining measurements of varied backing materials with a weighted average is therefore a powerful approach and has pushed conventional DSAM techniques to their lowest limit. 



% % uncomment the following lines,
% if using chapter-wise bibliography
%
% \bibliographystyle{ndnatbib}
% \bibliography{example}
