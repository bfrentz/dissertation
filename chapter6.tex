%
% Modified by Megan Patnott
% Last Change: Jan 18, 2013
%
%%%%%%%%%%%%%%%%%%%%%%%%%%%%%%%%%%%%%%%%%%%%%%%%%%%%%%%%%%%%%%%%%%%%%%%%
%
% Modified by Sameer Vijay
% Last Change: Tue Jul 26 2005 13:00 CEST
%
%%%%%%%%%%%%%%%%%%%%%%%%%%%%%%%%%%%%%%%%%%%%%%%%%%%%%%%%%%%%%%%%%%%%%%%%
%
% Sample Notre Dame Thesis/Dissertation
% Using Donald Peterson's ndthesis classfile
%
% Written by Jeff Squyres and Don Peterson
%
% Provided by the Information Technology Committee of
%   the Graduate Student Union
%   http://www.gsu.nd.edu/
%
% Nothing in this document is serious except the format.  :-)
%
% If you have any suggestions, comments, questions, please send e-mail
% to: ndthesis@gsu.nd.edu
%
%%%%%%%%%%%%%%%%%%%%%%%%%%%%%%%%%%%%%%%%%%%%%%%%%%%%%%%%%%%%%%%%%%%%%%%%


%
% Chapter 6
%

\chapter{Results and conclusions}
\label{chap: conclusions}



In this work, we measured the lifetimes of the excited states in $^{15}$O at $E_{x}$ = 5.18 MeV, 6.17 MeV, and 6.79 MeV. The $^{14}$N$(p,\gamma)^{15}$O reaction was used to populate the excited states of $^{15}$O. The nitrogen targets were made by implantation on backings of Mo, Ta, and W in order to examine the effect of different backing materials for the lifetime determination. The Doppler shift of the $\gamma$-rays emitted by the decaying recoils were measured using the different targets and at seven different angles. A Monte Carlo simulation was applied to reproduce the experimental shifts and extract the lifetimes from the measured attenuation factors. By using multiple implanted targets of different backings, we were able to take a weighted average of our measurements to reduce the overall systematic uncertainty. Additionally, the Monte Carlo approach allowed us to recreate the depth profile of implanted targets with a high degree of accuracy, making the subsequent analysis based on the target composition more robust. The simulation also propagates uncertainties throughout every step, allowing it to reflect the experimental conditions more accurately. This is an improvement over previous measurements and their treatment of their targets.


\begin{table}[]
\thisfloatpagestyle{plain}
\caption{MEASURED LIFETIMES AND COMPARISON WITH PREVIOUS MEASUREMENTS}
\begin{center}
\begin{threeparttable}
\begin{tabular}{lllll}
\toprule
$E_{x}$ (keV) & Present       & Ref. \cite{Bertone2001}  & Ref. \cite{Schurmann2008} & Ref. \cite{Galinski2014} \\
\midrule
5181          & $7.5 \pm 3.0$ & 9.67$^{+1.34}_{-1.24}$ & 8.40$\pm$1.00            & -                       \\
6172          & $0.7 \pm 0.5$ & 2.10$^{+1.33}_{-1.32}$  & $< 0.77$                 & $< 2.5$                 \\
6793          & $0.6 \pm 0.4$ & 1.60$^{+0.75}_{-0.72}$  & $< 0.77$                 & $< 1.8$    \\ \bottomrule
\end{tabular}
\begin{tablenotes}
\small 
\item A summary of the lifetimes (all given in fs) for the excited states in $^{15}$O determined in this work and how they compare with those of previous measurements. 
\end{tablenotes}
\end{threeparttable}
\label{table: lifetimesConclusion}
\end{center}
\end{table}

The results show no evidence of systematic variations with previous measurements arising from the choice of backing materials. This work shows a larger uncertainty for the lifetime of the 5.18 MeV state but agrees within the uncertainties of the previous measurements. For the other transitions at 6.17 MeV and 6.79 MeV, the present measurement agrees well with the values reported in previous works. Our work represents another finite measurement for the lifetime of the 6.79 MeV state, like reported by \citet{Bertone2001}. This work, however, is in agreement with the limits provided by \citet{Schurmann2008} and \citet{Galinski2014} and provides even more stringent constraints on the level lifetimes. The discrepancies in previous measurements were resolved in this measurement with three different backings. All of our reported lifetimes are given in comparison with those of previous measurements in Table \ref{table: lifetimesConclusion}.


\section{Reaction rates}
\label{sec: rates}



\section{$^{14}$N$\left( p,\gamma \right) ^{15}$O reaction outlook}
\label{sec: outlook}

At this point, there isn't much else that we can say or do. A better, longer measurement of the capture to ground state at really low energies would be the best. Beyond that, much comes down to how the data is interpreted. The region immediately above the $E_{p}$ = 278 keV resonance in the $^{14}$N$\left( p,\gamma \right) ^{15}$O reaction is immensely difficult due to the diffusion of nitrogen into the backing. With how strong the resonance is when compared to the direct capture for the ground state, any amount of nitrogen that diffuses into the backing will cause an elevation in the data at energies above the resonance. Therefore, if the data is cut out in the region of $\sim 300 - 400$ keV and ignored, the fit might be more accurate and provide a higher quality understanding of the behavior at low energy. 

In terms of doing a better measurement of the important lifetimes, it requires more data. Taking more measurements with even more varying targets would allow for an even more precise measurements (especially when taken in aggregate). However, as was seen, it would take a drastic reduction in the lifetime uncertainty to provide a meaningful constraint on the low-energy reaction cross-section. Such a drastic reduction is likely to be impossible due to the lifetime's proximity to the technique's limit. So, with conventional techniques maybe just measure with even more targets and take a better weighted average is the only improvement to be made? However, a measurement with a state-of-the-art detection array with gamma tracking, like the AGATA Demonstrator or (the MSU ONE), would be the ultimate tool in reducing the lifetime uncertainty by making the angular measurements extremely precise for nearly every single gamma produced in the reaction. 



% % uncomment the following lines,
% if using chapter-wise bibliography
%
% \bibliographystyle{ndnatbib}
% \bibliography{example}
